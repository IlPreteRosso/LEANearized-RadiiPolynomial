% In this file you should put the actual content of the blueprint.
% It will be used both by the web and the print version.
% It should *not* include the \begin{document}
%
% If you want to split the blueprint content into several files then
% the current file can be a simple sequence of \input. Otherwise It
% can start with a \section or \chapter for instance.

\tableofcontents
\pagebreak

\section{Radii polynomials in finite dimensions}\label{sec:radii-poly-fd}

%----------------------------------------------------
\begin{definition}[Finite-dimensional data]\label{def:data}
  \lean{RP.Data}
  Fix a norm $\|\cdot\|$ on $\mathbb R^n$ (or $\mathbb C^n$). Let $f:\mathbb R^n\to\mathbb R^n$ be $C^1$,
  $\bar x\in\mathbb R^n$ a numerical approximation to a zero of $f$, and
  $A\in M_n(\mathbb R)$ (typically $A\approx Df(\bar x)^{-1}$).
\end{definition}

%----------------------------------------------------
\begin{definition}[Newton-like map]\label{def:T}
  \lean{RP.T}
  \uses{def:data}
  Define $T:\mathbb R^n\to\mathbb R^n$ by $T(x)=x-A f(x)$.
\end{definition}

%----------------------------------------------------
\begin{definition}[A priori bounds $Y_0,Z_0,Z_2$]\label{def:bounds}
  \lean{RP.Bounds}
  \uses{def:data,def:T}
  Choose nonnegative $Y_0,Z_0$ and a function $Z_2:(0,\infty)\to[0,\infty)$ such that
  for all $r>0$ and all $c\in B_r(\bar x)$,
  \[
    \|A f(\bar x)\|\le Y_0,\qquad
    \|I-ADf(\bar x)\|\le Z_0,\qquad
    \|A(Df(c)-Df(\bar x))\|\le Z_2(r)\,r.
  \]
\end{definition}

%----------------------------------------------------
\begin{definition}[Radii polynomial]\label{def:rpoly}
  \lean{RP.RadiiPolynomial}
  \uses{def:bounds}
  For $r>0$, set
  \[
    p(r):=Z_2(r)\,r^2-(1-Z_0)\,r+Y_0.
  \]
\end{definition}

%----------------------------------------------------
\begin{lemma}[Derivative bound]\label{lem:DT-bound}
  \lean{RP.DT_bound}
  \uses{def:T,def:bounds}
  For every $c\in B_r(\bar x)$,
  \[
    DT(c)=I-ADf(c),\qquad
    \|DT(c)\|\le Z_0+Z_2(r)\,r =: Z(r).
  \]
\end{lemma}
\begin{proof}
  \leanok % (mark proof as fully formalized once ported)
  Immediate from $DT=I-ADf$ and the $Z_0,Z_2$ hypotheses.
\end{proof}

%----------------------------------------------------
\begin{lemma}[Ball invariance and contraction criterion]\label{lem:fixed-point-radii}
  \lean{RP.FixedPointRadii}
  \uses{def:T,lem:DT-bound}
  If $r>0$ satisfies $(Z(r)-1)r+Y_0<0$, then $T(B_r(\bar x))\subseteq B_r(\bar x)$ and
  $T$ is a contraction on $B_r(\bar x)$ with Lipschitz constant $\le Z(r)<1$.
\end{lemma}
\begin{proof}
  \leanok
  Use the mean value inequality for $T$ together with $\|T(\bar x)-\bar x\|\le Y_0$ and $\sup_{c\in B_r}\|DT(c)\|\le Z(r)$.
\end{proof}

%----------------------------------------------------
\begin{theorem}[Radii polynomials in finite dimensions]\label{thm:radii-poly-fd}
  \lean{RP.RadiiPolynomialFiniteDim}
  \uses{def:rpoly,lem:DT-bound,lem:fixed-point-radii}
  If there exists $r_0>0$ with $p(r_0)<0$, then:
  \begin{enumerate}[label=(\roman*)]
    \item $T$ maps $B_{r_0}(\bar x)$ into itself and is a contraction there;
    \item there exists a unique $\tilde x\in B_{r_0}(\bar x)$ with $T(\tilde x)=\tilde x$;
    \item $f(\tilde x)=0$, and $\|I-ADf(\tilde x)\|<1$ hence $Df(\tilde x)$ is invertible.
  \end{enumerate}
\end{theorem}
\begin{proof}
  \leanok
  From $p(r_0)<0$ we get $Z(r_0)<1$ and $(Z(r_0)-1)r_0+Y_0<0$.
  Apply Lemma~\ref{lem:fixed-point-radii} with $r=r_0$ to obtain (i) and (ii) by the contraction mapping theorem.
  Then $T(\tilde x)=\tilde x$ implies $A f(\tilde x)=0$, hence $f(\tilde x)=0$ since $A$ is injective.
  Finally $\|I-ADf(\tilde x)\|<1$ implies $Df(\tilde x)$ invertible (Neumann series).
\end{proof}
