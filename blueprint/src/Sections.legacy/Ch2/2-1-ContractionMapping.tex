% blueprint/src/Sections/Ch2/2-1-ContractionMapping.tex

\section{Contraction Mapping Theorem}\label{sec:contraction}

The contraction mapping theorem is a fundamental tool for proving existence and uniqueness 
of fixed points. This section establishes the basic definitions and the classical theorem.

\begin{definition}[Complete metric space]\label{def:complete_metric_space}
  \lean{CompleteSpace}
  \leanok
  A metric space $(X, d)$ is \textit{complete} if every Cauchy sequence converges in $X$, 
  i.e., if given any $\varepsilon > 0$ there exists $N(\varepsilon)$ such that 
  $n, m > N(\varepsilon)$ implies $d(x_n, x_m) < \varepsilon$, then there is some 
  $y \in X$ with $\lim_{n\to\infty} x_n = y$.
\end{definition}

\begin{definition}[Contraction mapping]\label{def:contraction}
  \lean{ContractingWith}
  \leanok
  Let $(X, d)$ be a metric space. A function $T: X \to X$ is a \textit{contraction} if 
  there is a number $\kappa \in [0, 1)$, called a \textit{contraction constant}, such that
  \begin{equation*}
  d(T(x), T(y)) \leq \kappa \cdot d(x, y)
\end{equation*}
  for all $x, y \in X$.
\end{definition}

\begin{theorem}[Contraction Mapping Theorem]\label{thm:contraction_mapping}
  \lean{ContractingWith.exists_fixedPoint'}
  \leanok
  Let $(X, d)$ be a complete metric space. If $T: X \to X$ is a contraction with 
  contraction constant $\kappa$, then there exists a unique fixed point $\tilde{x} \in X$ 
  of $T$. Furthermore, $\tilde{x}$ is globally attracting, and for any $x \in X$,
  \begin{equation*}
  d(T^n(x), \tilde{x}) \leq \frac{\kappa^n}{1-\kappa} d(T(x), x).
\end{equation*}
\end{theorem}

\begin{remark}\label{rem:contraction_rate}
  Given a contraction mapping $T: X \to X$, the rate at which points in $X$ converge 
  to the globally attracting fixed point $\tilde{x}$ is determined by $\kappa$. 
  In particular, the smaller $\kappa$ is, the faster iterates under $T$ converge to $\tilde{x}$.
\end{remark}
