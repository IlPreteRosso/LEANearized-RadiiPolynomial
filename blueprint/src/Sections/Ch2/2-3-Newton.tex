% blueprint/src/Sections/Ch2/2-3-Newton.tex

\section{Newton's Method}\label{sec:newton}

Newton's method transforms the problem of finding zeros into the problem of finding 
fixed points. This section establishes the fundamental equivalence and introduces 
Newton-like operators.

\begin{definition}[Newton-like map]\label{def:newton_like_map}
  \lean{NewtonLikeMap}
  \leanok
  Let $E, F$ be Banach spaces, $f: E \to F$ a function, and $A: F \to E$ a continuous 
  linear map. The \textit{Newton-like map} is defined by
  \begin{equation*}
    T(x) = x - A(f(x)).
  \end{equation*}
\end{definition}

\begin{proposition}[Fixed points $\iff$ Zeros]\label{prop:fixed_point_iff_zero}
  \lean{fixedPoint_injective_iff_zero}
  \leanok
  \uses{def:newton_like_map}
  Let $f: E \to F$ and $A: F \to E$ be an injective linear map. 
  Let $T(x) = x - A(f(x))$ be the Newton-like operator. Then:
  \begin{equation*}
    T(x) = x \quad \iff \quad f(x) = 0.
  \end{equation*}
\end{proposition}

\begin{proof}
  \leanok
  First direction ($T(x) = x \Rightarrow f(x) = 0$):
  
  If $T(x) = x$, then $x - A(f(x)) = x$, which gives $A(f(x)) = 0$. 
  Since $A$ is linear, $A(0) = 0$. By injectivity of $A$, we have $A(f(x)) = A(0)$ 
  implies $f(x) = 0$.
  
  Second direction ($f(x) = 0 \Rightarrow T(x) = x$):
  
  If $f(x) = 0$, then $T(x) = x - A(f(x)) = x - A(0) = x - 0 = x$.
\end{proof}

\begin{definition}[Nondegenerate zero]\label{def:nondegenerate_zero}
  Let $f \in C^1(U, \mathbb{R}^n)$ where $U \subset \mathbb{R}^n$ is an open set. 
  A point $\tilde{x} \in U$ is a \textit{nondegenerate zero} of $f$ if $f(\tilde{x}) = 0$ 
  and $\text{Df}(\tilde{x})$ is invertible.
\end{definition}

\begin{definition}[Classical Newton operator]\label{def:classical_newton}
  If $f \in C^2(U, \mathbb{R}^n)$ where $U \subset \mathbb{R}^n$ is an open set, 
  the \textit{classical Newton operator} is given by
  \begin{equation*}
    T(x) := x - (\text{Df}(x))^{-1} f(x).
  \end{equation*}
\end{definition}

\begin{remark}\label{rem:newton_derivative_at_zero}
  If $\tilde{x}$ is a nondegenerate zero of $f \in C^2$, then the derivative of the 
  classical Newton operator satisfies
  \begin{equation*}
    \text{DT}(\tilde{x}) = I - (\text{Df}(\tilde{x}))^{-1} \text{Df}(\tilde{x}) = 0,
  \end{equation*}
  making $T$ a strong contraction near $\tilde{x}$.
\end{remark}
