% blueprint/src/Sections/Ch2/2-3-Newton.tex

\section{Newton's Method}\label{sec:newton}

Newton's method transforms the problem of finding zeros into the problem of finding 
fixed points. This section establishes the fundamental equivalence and introduces 
Newton-like operators.

\begin{definition}[Newton-like map]\label{def:newton_like_map}
  \lean{NewtonLikeMap}
  \leanok
  Let $E, F$ be Banach spaces, $f: E \to F$ a function, and $A: F \to E$ a continuous 
  linear map. The \textit{Newton-like map} is defined by
  \begin{equation*}
  T(x) = x - A(f(x)).
  \end{equation*}
\end{definition}

\begin{proposition}[Fixed points $\iff$ Zeros (Proposition 2.3.1)]\label{prop:fixed_point_iff_zero}
  \lean{fixedPoint_injective_iff_zero}
  \leanok
  \uses{def:newton_like_map}
  Let $E, F$ be vector spaces, $f: E \to F$, and $A: F \to E$ an injective linear map. 
  Let $T(x) = x - A(f(x))$ be the Newton-like operator. Then:
  \begin{equation*}
  T(x) = x \quad \iff \quad f(x) = 0.
  \end{equation*}
\end{proposition}

\begin{proof}
  \leanok
  ($\Rightarrow$) If $T(x) = x$, then $x - A(f(x)) = x$, so $A(f(x)) = 0$. 
  Since $A$ is linear, $A(0) = 0$. By injectivity of $A$, $A(f(x)) = A(0)$ implies $f(x) = 0$.
  
  ($\Leftarrow$) If $f(x) = 0$, then $T(x) = x - A(0) = x - 0 = x$.
\end{proof}