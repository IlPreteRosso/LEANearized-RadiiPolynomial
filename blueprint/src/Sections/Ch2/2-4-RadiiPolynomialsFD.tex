% blueprint/src/Sections/Ch2/2-4-RadiiPolynomialsFD.tex

\section{Radii Polynomials in Finite Dimensions}\label{sec:radii_poly_fd}

This section develops the radii polynomial method for proving existence of zeros in 
finite dimensions. The method provides explicit domains of existence and uniqueness.

\begin{theorem}[Fixed point theorem with radii polynomial]\label{thm:fixed_point_radii}
  \uses{def:contraction, cor:mean_value_inequality}
  Consider a map $T \in C^1(\mathbb{R}^n, \mathbb{R}^n)$ and let $\bar{x} \in \mathbb{R}^n$. 
  Let $Y_0 \geq 0$ and $Z: (0,\infty) \to [0,\infty)$ be a non-negative function satisfying
  \begin{align}
    \|T(\bar{x}) - \bar{x}\| &\leq Y_0 \label{eq:2.10}\\
    \|\text{DT}(c)\| &\leq Z(r), \quad \text{for all } c \in \overline{B}_r(\bar{x}) 
    \text{ and all } r > 0. \label{eq:2.11}
  \end{align}
  Define
  \[
    p(r) := (Z(r) - 1)r + Y_0.
  \]
  If there exists $r_0 > 0$ such that $p(r_0) < 0$, then there exists a unique 
  $\tilde{x} \in \overline{B}_{r_0}(\bar{x})$ such that $T(\tilde{x}) = \tilde{x}$.
\end{theorem}

\begin{definition}[Radii polynomial]\label{def:radii_polynomial}
  \lean{radiiPolynomial}
  \leanok
  Given constants $Y_0, Z_0 \geq 0$ and a function $Z_2: (0,\infty) \to [0,\infty)$, 
  the \emph{radii polynomial} is defined by
  \[
    p(r) := Z_2(r)r^2 - (1-Z_0)r + Y_0.
  \]
\end{definition}

\begin{definition}[Combined bound]\label{def:Z_bound}
  \lean{Z_bound}
  \leanok
  The \emph{combined bound} is defined as
  \[
    Z(r) := Z_0 + Z_2(r) \cdot r.
  \]
\end{definition}

\begin{lemma}[Alternative form of radii polynomial]\label{lem:radii_poly_alt_form}
  \lean{radiiPolynomial_alt_form}
  \leanok
  \uses{def:radii_polynomial, def:Z_bound}
  The radii polynomial can be rewritten as
  \[
    p(r) = (Z(r) - 1)r + Y_0.
  \]
\end{lemma}

\begin{lemma}[Polynomial negativity implies contraction]\label{lem:poly_neg_implies_contraction}
  \lean{radii_poly_neg_implies_Z_bound_lt_one}
  \leanok
  \uses{def:radii_polynomial, def:Z_bound}
  If $Y_0 \geq 0$, $r_0 > 0$, and $p(r_0) < 0$, then $Z(r_0) < 1$.
\end{lemma}

\begin{theorem}[Radii polynomial theorem in finite dimensions]\label{thm:radii_poly_fd}
  \lean{simple_radii_polynomial_theorem_same_space}
  \leanok
  \uses{def:newton_like_map, def:radii_polynomial, prop:fixed_point_iff_zero}
  Consider $f \in C^1(\mathbb{R}^n, \mathbb{R}^n)$. Let $\bar{x} \in \mathbb{R}^n$ and 
  $A \in M_n(\mathbb{R})$. Let $Y_0$ and $Z_0$ be non-negative constants and 
  $Z_2: (0,\infty) \to [0,\infty)$ be a non-negative function satisfying
  \begin{align}
    \|Af(\bar{x})\| &\leq Y_0 \label{eq:2.14}\\
    \|I - A\text{Df}(\bar{x})\| &\leq Z_0 \label{eq:2.15}\\
    \|A[\text{Df}(c) - \text{Df}(\bar{x})]\| &\leq Z_2(r) \cdot r, 
    \quad \text{for all } c \in \overline{B}_r(\bar{x}) \text{ and all } r > 0. \label{eq:2.16}
  \end{align}
  Define
  \[
    p(r) := Z_2(r)r^2 - (1-Z_0)r + Y_0.
  \]
  If there exists $r_0 > 0$ such that $p(r_0) < 0$, then there exists a unique 
  $\tilde{x} \in \overline{B}_{r_0}(\bar{x})$ satisfying $f(\tilde{x}) = 0$ and 
  $\text{Df}(\tilde{x})$ is invertible (hence $\tilde{x}$ is a nondegenerate zero).
\end{theorem}

\begin{proof}
  \leanok
  \uses{thm:fixed_point_radii, lem:poly_neg_implies_contraction}
  Define the Newton-like mapping $T(x) = x - Af(x)$.
  
  \textbf{Step 1: Verify bounds.}
  We have $\|T(\bar{x}) - \bar{x}\| = \|Af(\bar{x})\| \leq Y_0$ by \eqref{eq:2.14}.
  
  For $c \in \overline{B}_{r_0}(\bar{x})$,
  \begin{align*}
    \|\text{DT}(c)\| &= \|I - A\text{Df}(c)\|\\
    &\leq \|I - A\text{Df}(\bar{x})\| + \|A[\text{Df}(\bar{x}) - \text{Df}(c)]\|\\
    &\leq Z_0 + Z_2(r_0) \cdot r_0 \quad \text{by \eqref{eq:2.15} and \eqref{eq:2.16}}\\
    &= Z(r_0).
  \end{align*}
  
  \textbf{Step 2: Apply fixed point theorem.}
  Since $p(r_0) = (Z(r_0) - 1)r_0 + Y_0 < 0$ by assumption, 
  Theorem~\ref{thm:fixed_point_radii} gives a unique fixed point 
  $\tilde{x} \in \overline{B}_{r_0}(\bar{x})$ with $T(\tilde{x}) = \tilde{x}$.
  
  \textbf{Step 3: Show $A$ is invertible.}
  From $p(r_0) < 0$ we get $Z(r_0) < 1$ by Lemma~\ref{lem:poly_neg_implies_contraction}. 
  In particular, $\|I - A\text{Df}(\bar{x})\| \leq Z_0 \leq Z(r_0) < 1$, so 
  $A\text{Df}(\bar{x})$ is invertible by the Neumann series. Since $\text{Df}(\bar{x})$ 
  is square and finite-dimensional, this implies $A$ is invertible.
  
  \textbf{Step 4: Convert fixed point to zero.}
  By Proposition~\ref{prop:fixed_point_iff_zero} and invertibility of $A$, 
  $T(\tilde{x}) = \tilde{x}$ implies $f(\tilde{x}) = 0$.
  
  \textbf{Step 5: Show $\text{Df}(\tilde{x})$ is invertible.}
  Since $\tilde{x} \in \overline{B}_{r_0}(\bar{x})$, we have 
  $\|\text{DT}(\tilde{x})\| = \|I - A\text{Df}(\tilde{x})\| \leq Z(r_0) < 1$. 
  By the Neumann series, $A\text{Df}(\tilde{x})$ is invertible, which implies 
  $\text{Df}(\tilde{x})$ is invertible (using that $A$ is invertible).
\end{proof}

\begin{definition}[Existence interval]\label{def:existence_interval}
  Consider a radii polynomial $p(r) = Z_2(r)r^2 - (1-Z_0)r + Y_0$ for a function 
  $f: \mathbb{R}^n \to \mathbb{R}^n$ where $Z_2(r)$ is a polynomial with non-negative 
  coefficients (or a non-decreasing function of $r$). The maximal interval 
  $(r_-, r_+) \subset (0,\infty)$ over which $p(r) < 0$ is called the 
  \emph{existence interval} for the radii polynomial $p$, denoted by $\text{EI}(p)$.
\end{definition}

\begin{remark}\label{rem:existence_interval_interpretation}
  If the existence interval for a radii polynomial is nonempty, then:
  \begin{itemize}
    \item $r_-$ provides tight bounds on the location of $\tilde{x}$
    \item $r_+$ provides information about the domain of isolation of $\tilde{x}$
    \item The unique zero $\tilde{x}$ satisfies $\tilde{x} \in \overline{B}_r(\bar{x})$ 
          for all $r \in \text{EI}(p)$
  \end{itemize}
\end{remark}
