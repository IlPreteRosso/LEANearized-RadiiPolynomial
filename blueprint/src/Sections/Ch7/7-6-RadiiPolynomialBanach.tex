% blueprint/src/Sections/Ch7/7-6-RadiiPolynomialBanach.tex

\section{Radii Polynomial Approach on Banach Spaces}\label{sec:radii_poly_banach}

This section extends the radii polynomial method to infinite-dimensional Banach spaces, 
allowing for maps between potentially different spaces $E$ and $F$.

\subsection{Banach Space Setup}\label{subsec:banach_setup}

We work with two Banach spaces $E$ and $F$ over $\mathbb{R}$. For each space 
$X \in \{E, F\}$:
\begin{enumerate}
  \item \texttt{NormedAddCommGroup X}: $X$ has a norm satisfying definiteness, 
        symmetry, triangle inequality
  \item \texttt{NormedSpace $\mathbb{R}$ X}: The norm is compatible with scalar multiplication
  \item \texttt{CompleteSpace X}: Every Cauchy sequence converges 
        (crucial for fixed point theorems)
\end{enumerate}

This framework supports:
\begin{itemize}
  \item Fréchet derivatives (via the norm structure)
  \item Fixed point theorems (via completeness)
  \item Mean Value Theorem (via the metric structure)
  \item Linear operator theory (via the vector space structure)
\end{itemize}

\subsection{Neumann Series and Operator Invertibility}\label{subsec:neumann_series}

The Neumann series provides a constructive way to show operators close to the identity 
are invertible.

\begin{theorem}[Neumann series invertibility]\label{thm:neumann_series}
  \lean{isUnit_of_norm_sub_id_lt_one}
  \leanok
  Let $E$ be a Banach space and $B: E \to E$ a continuous linear operator. 
  If $\|I_E - B\| < 1$, then $B$ is invertible (a unit in the multiplicative sense).
\end{theorem}

\begin{lemma}[Explicit two-sided inverse]\label{lem:explicit_inverse}
  \lean{invertible_of_norm_sub_id_lt_one}
  \leanok
  \uses{thm:neumann_series}
  If $\|I_E - B\| < 1$ for $B: E \to E$, then there exists $B^{-1}: E \to E$ such that
  \begin{equation*}
    B \circ B^{-1} = I_E \quad \text{and} \quad B^{-1} \circ B = I_E.
  \end{equation*}
\end{lemma}

\begin{lemma}[Composition form]\label{lem:invertible_comp_form}
  \lean{invertible_comp_form}
  \leanok
  \uses{lem:explicit_inverse}
  If $\|I_E - B\| < 1$, then there exists $B^{-1}$ such that
  \begin{equation*}
    B.comp(B^{-1}) = I_E \quad \text{and} \quad B^{-1}.comp(B) = I_E.
  \end{equation*}
\end{lemma}

\subsection{Newton-Like Operators for E to F Maps}\label{subsec:newton_ef}

\begin{definition}[Newton-like map for E to F]\label{def:newton_ef}
  \lean{NewtonLikeMap}
  \leanok
  For a function $f: E \to F$ and an approximate inverse $A: F \to E$, the 
  \textit{Newton-like map} is
  \begin{equation*}
    T(x) = x - A(f(x)).
  \end{equation*}
  Note that $T: E \to E$ even though $f$ maps between different spaces.
\end{definition}

\begin{proposition}[Fixed points ⟺ Zeros for E to F]\label{prop:fixed_zero_ef}
  \lean{fixedPoint_injective_iff_zero}
  \leanok
  \uses{def:newton_ef}
  Let $f: E \to F$ and $A: F \to E$ be injective. Then for the Newton-like operator 
  $T(x) = x - A(f(x))$:
  \begin{equation*}
    T(x) = x \quad \iff \quad f(x) = 0.
  \end{equation*}
  This holds even when $E \neq F$; injectivity of $A$ is sufficient.
\end{proposition}

\subsection{Radii Polynomial Definitions}\label{subsec:radii_poly_defs_banach}

\begin{definition}[General radii polynomial]\label{def:general_radii_poly}
  \lean{generalRadiiPolynomial}
  \leanok
  For constants $Y_0, Z_0, Z_1 \geq 0$ and function $Z_2: (0,\infty) \to [0,\infty)$, 
  the \textit{general radii polynomial} is
  \begin{equation*}
    p(r) := Z_2(r)r^2 - (1 - Z_0 - Z_1)r + Y_0.
  \end{equation*}
\end{definition}

\begin{definition}[Combined bound (general case)]\label{def:Z_bound_general}
  \lean{Z_bound_general}
  \leanok
  The \textit{combined bound} is
  \begin{equation*}
    Z(r) := Z_0 + Z_1 + Z_2(r) \cdot r.
  \end{equation*}
\end{definition}

\begin{lemma}[Alternative form (general)]\label{lem:general_poly_alt_form}
  \lean{generalRadiiPolynomial_alt_form}
  \leanok
  \uses{def:general_radii_poly, def:Z_bound_general}
  The general radii polynomial can be rewritten as
  \begin{equation*}
    p(r) = (Z(r) - 1)r + Y_0.
  \end{equation*}
\end{lemma}

\begin{definition}[Simple radii polynomial]\label{def:simple_radii_poly_banach}
  \lean{simpleRadiiPolynomial}
  \leanok
  For $Y_0 \geq 0$ and $Z: (0,\infty) \to [0,\infty)$, the \textit{simple radii polynomial} is
  \begin{equation*}
    p(r) := (Z(r) - 1)r + Y_0.
  \end{equation*}
\end{definition}

\subsection{Operator Bounds}\label{subsec:operator_bounds}

\begin{lemma}[Y₀ bound for Newton operator]\label{lem:newton_Y_bound}
  \lean{newton_operator_Y_bound}
  \leanok
  \uses{def:newton_ef}
  If $\|A(f(\bar{x}))\| \leq Y_0$, then for $T(x) = x - A(f(x))$:
  \begin{equation*}
    \|T(\bar{x}) - \bar{x}\| \leq Y_0.
  \end{equation*}
\end{lemma}

\begin{lemma}[Derivative of Newton operator]\label{lem:newton_fderiv}
  \lean{newton_operator_fderiv}
  \leanok
  \uses{def:newton_ef}
  For $T(x) = x - A(f(x))$ where $f: E \to F$ is differentiable:
  \begin{equation*}
    \text{DT}(x) = I_E - A \circ \text{Df}(x).
  \end{equation*}
\end{lemma}

\begin{lemma}[General derivative bound]\label{lem:newton_deriv_bound_general}
  \lean{newton_operator_derivative_bound_general}
  \leanok
  \uses{lem:newton_fderiv, def:Z_bound_general}
  Suppose for all $c \in \overline{B}_r(\bar{x})$:
  \begin{align}
    \|I_E - A \circ A^\dagger\| &\leq Z_0 \label{eq:7.31}\\
    \|A \circ (A^\dagger - \text{Df}(\bar{x}))\| &\leq Z_1 \label{eq:7.32}\\
    \|A \circ (\text{Df}(c) - \text{Df}(\bar{x}))\| &\leq Z_2(r) \cdot r \label{eq:7.33}
  \end{align}
  Then for $T(x) = x - A(f(x))$ and $c \in \overline{B}_r(\bar{x})$:
  \begin{equation*}
    \|\text{DT}(c)\| \leq Z_0 + Z_1 + Z_2(r) \cdot r = Z(r).
  \end{equation*}
\end{lemma}

\begin{proof}
  \leanok
  For $c \in \overline{B}_r(\bar{x})$, decompose using $A^\dagger$:
  \begin{align*}
    \text{DT}(c) &= I_E - A \circ \text{Df}(c)\\
    &= [I_E - A \circ A^\dagger] + A \circ [A^\dagger - \text{Df}(\bar{x})] 
       + A \circ [\text{Df}(\bar{x}) - \text{Df}(c)].
  \end{align*}
  Apply triangle inequality and the three bounds \eqref{eq:7.31}, \eqref{eq:7.32}, 
  \eqref{eq:7.33}:
  \begin{equation*}
    \|\text{DT}(c)\| \leq Z_0 + Z_1 + Z_2(r) \cdot r.
  \end{equation*}
\end{proof}

\begin{lemma}[Simple derivative bound]\label{lem:newton_deriv_bound_simple}
  \lean{newton_operator_derivative_bound_simple}
  \leanok
  \uses{lem:newton_fderiv}
  When $A^\dagger = \text{Df}(\bar{x})$ (so $Z_1 = 0$), for all 
  $c \in \overline{B}_r(\bar{x})$:
  \begin{align}
    \|I_E - A \circ \text{Df}(\bar{x})\| &\leq Z_0 \label{eq:2.15b}\\
    \|A \circ (\text{Df}(c) - \text{Df}(\bar{x}))\| &\leq Z_2(r) \cdot r \label{eq:2.16b}
  \end{align}
  imply $\|\text{DT}(c)\| \leq Z_0 + Z_2(r) \cdot r$.
\end{lemma}

\subsection{Helper Lemmas}\label{subsec:helper_lemmas}

\begin{lemma}[Closed balls are complete]\label{lem:closed_ball_complete}
  \lean{isComplete_closedBall}
  \leanok
  In a complete space $E$, closed balls $\overline{B}_r(x)$ are complete.
\end{lemma}

\begin{lemma}[Extended distance is finite]\label{lem:edist_finite}
  \lean{edist_ne_top_of_normed}
  \leanok
  In normed spaces, extended distance is always finite: $d_{\text{ext}}(x,y) \neq \top$.
\end{lemma}

\begin{lemma}[Constructing derivative inverse]\label{lem:construct_deriv_inverse}
  \lean{construct_derivative_inverse}
  \leanok
  \uses{thm:neumann_series}
  If $A: F \to E$ is injective and $\|I_E - A \circ B\| < 1$ for $B: E \to F$, 
  then $B$ is invertible with inverse $B^{-1} = (A \circ B)^{-1} \circ A$.
\end{lemma}

\begin{lemma}[Ball mapping property]\label{lem:maps_ball_to_itself}
  \lean{simple_maps_closedBall_to_itself}
  \leanok
  \uses{cor:mean_value_inequality}
  Given $T: E \to E$ differentiable with:
  \begin{itemize}
    \item $\|T(\bar{x}) - \bar{x}\| \leq Y_0$
    \item $\|\text{DT}(c)\| \leq Z(r_0)$ for all $c \in \overline{B}_{r_0}(\bar{x})$
    \item $p(r_0) = (Z(r_0) - 1)r_0 + Y_0 < 0$
  \end{itemize}
  Then $T: \overline{B}_{r_0}(\bar{x}) \to \overline{B}_{r_0}(\bar{x})$.
\end{lemma}

\begin{proof}
  \leanok
  Let $x \in \overline{B}_{r_0}(\bar{x})$. From $p(r_0) < 0$:
  \begin{equation*}
    Z(r_0) \cdot r_0 + Y_0 < r_0.
  \end{equation*}
  
  By Mean Value Theorem:
  \begin{equation*}
    \|T(x) - T(\bar{x})\| \leq Z(r_0) \cdot \|x - \bar{x}\| \leq Z(r_0) \cdot r_0.
  \end{equation*}
  
  By triangle inequality:
  \begin{align*}
    \|T(x) - \bar{x}\| &\leq \|T(x) - T(\bar{x})\| + \|T(\bar{x}) - \bar{x}\|\\
    &\leq Z(r_0) \cdot r_0 + Y_0 < r_0.
  \end{align*}
  Therefore $T(x) \in \overline{B}_{r_0}(\bar{x})$.
\end{proof}

\subsection{Main Theorems}\label{subsec:main_theorems}

\begin{theorem}[General Fixed Point Theorem (Theorem 7.6.1)]\label{thm:general_fixed_point}
  \lean{general_fixed_point_theorem}
  \leanok
  \uses{def:simple_radii_poly_banach, lem:maps_ball_to_itself, thm:contraction_mapping}
  Let $T: E \to E$ be Fréchet differentiable and $\bar{x} \in E$. Suppose:
  \begin{align}
    \|T(\bar{x}) - \bar{x}\| &\leq Y_0 \label{eq:7.27}\\
    \|\text{DT}(x)\| &\leq Z(r) \quad \text{for all } x \in \overline{B}_r(\bar{x}) \label{eq:7.28}
  \end{align}
  Define $p(r) := (Z(r) - 1)r + Y_0$.
  
  If there exists $r_0 > 0$ such that $p(r_0) < 0$, then there exists a unique 
  $\tilde{x} \in \overline{B}_{r_0}(\bar{x})$ such that $T(\tilde{x}) = \tilde{x}$.
\end{theorem}

\begin{proof}
  \leanok
  \textbf{Step 1:} From $p(r_0) < 0$, we get $Z(r_0) < 1$ (contraction constant).
  
  \textbf{Step 2:} By Lemma~\ref{lem:maps_ball_to_itself}, 
  $T: \overline{B}_{r_0}(\bar{x}) \to \overline{B}_{r_0}(\bar{x})$.
  
  \textbf{Step 3:} $T$ restricted to $\overline{B}_{r_0}(\bar{x})$ is a contraction 
  with constant $Z(r_0) < 1$.
  
  \textbf{Step 4:} The closed ball is complete by Lemma~\ref{lem:closed_ball_complete}.
  
  \textbf{Step 5:} Apply Banach Fixed Point Theorem to get unique fixed point.
\end{proof}

\begin{theorem}[General Radii Polynomial Theorem (Theorem 7.6.2)]\label{thm:general_radii_poly}
  \lean{general_radii_polynomial_theorem}
  \leanok
  \uses{thm:general_fixed_point, prop:fixed_zero_ef, def:general_radii_poly}
  Let $E$ and $F$ be Banach spaces and $f: E \to F$ be Fréchet differentiable. 
  Suppose $\bar{x} \in E$, $A^\dagger: E \to F$, and $A: F \to E$ with $A$ injective. 
  Assume:
  \begin{align}
    \|A(f(\bar{x}))\| &\leq Y_0 \label{eq:7.30}\\
    \|I_E - A \circ A^\dagger\| &\leq Z_0 \label{eq:7.31b}\\
    \|A \circ [\text{Df}(\bar{x}) - A^\dagger]\| &\leq Z_1 \label{eq:7.32b}\\
    \|A \circ [\text{Df}(c) - \text{Df}(\bar{x})]\| &\leq Z_2(r) \cdot r 
    \quad \text{for } c \in \overline{B}_r(\bar{x}) \label{eq:7.33b}
  \end{align}
  Define
  \begin{equation*}
    p(r) := Z_2(r)r^2 - (1 - Z_0 - Z_1)r + Y_0.
  \end{equation*}
  
  If there exists $r_0 > 0$ such that $p(r_0) < 0$, then there exists a unique 
  $\tilde{x} \in \overline{B}_{r_0}(\bar{x})$ with $f(\tilde{x}) = 0$.
\end{theorem}

\begin{proof}
  \leanok
  \textbf{Step 1: Define Newton operator.}
  Let $T(x) = x - A(f(x))$. Then $T: E \to E$ is differentiable.
  
  \textbf{Step 2: Verify conditions of Theorem~\ref{thm:general_fixed_point}.}
  \begin{itemize}
    \item By Lemma~\ref{lem:newton_Y_bound}: $\|T(\bar{x}) - \bar{x}\| \leq Y_0$
    \item By Lemma~\ref{lem:newton_deriv_bound_general}: 
          $\|\text{DT}(c)\| \leq Z(r_0)$ for $c \in \overline{B}_{r_0}(\bar{x})$
    \item The polynomial condition: 
          $p(r_0) = (Z(r_0) - 1)r_0 + Y_0 < 0$ by Lemma~\ref{lem:general_poly_alt_form}
  \end{itemize}
  
  \textbf{Step 3: Apply Theorem~\ref{thm:general_fixed_point}.}
  Get unique $\tilde{x} \in \overline{B}_{r_0}(\bar{x})$ with $T(\tilde{x}) = \tilde{x}$.
  
  \textbf{Step 4: Convert to zero.}
  By Proposition~\ref{prop:fixed_zero_ef} with injectivity of $A$: $f(\tilde{x}) = 0$.
\end{proof}

\begin{theorem}[Simple Radii Polynomial (Same Space)]\label{thm:simple_radii_poly_same}
  \lean{simple_radii_polynomial_theorem_same_space}
  \leanok
  \uses{thm:general_fixed_point, lem:construct_deriv_inverse}
  Consider $f: E \to E$ Fréchet differentiable, $\bar{x} \in E$, and $A: E \to E$ 
  injective. Assume:
  \begin{align}
    \|A(f(\bar{x}))\| &\leq Y_0\\
    \|I_E - A \circ \text{Df}(\bar{x})\| &\leq Z_0\\
    \|A \circ [\text{Df}(c) - \text{Df}(\bar{x})]\| &\leq Z_2(r) \cdot r 
    \quad \text{for } c \in \overline{B}_r(\bar{x})
  \end{align}
  Define $p(r) := Z_2(r)r^2 - (1-Z_0)r + Y_0$.
  
  If $p(r_0) < 0$, then there exists unique $\tilde{x} \in \overline{B}_{r_0}(\bar{x})$ 
  with $f(\tilde{x}) = 0$ and $\text{Df}(\tilde{x})$ invertible.
\end{theorem}

\begin{proof}
  \leanok
  Apply Theorem~\ref{thm:general_fixed_point} to get fixed point $\tilde{x}$. 
  Convert to zero using Proposition~\ref{prop:fixed_zero_ef}.
  
  For invertibility: Since $\tilde{x} \in \overline{B}_{r_0}(\bar{x})$ and 
  $Z(r_0) < 1$, we have $\|I_E - A \circ \text{Df}(\tilde{x})\| < 1$. 
  Apply Lemma~\ref{lem:construct_deriv_inverse} to get $\text{Df}(\tilde{x})$ invertible.
\end{proof}
