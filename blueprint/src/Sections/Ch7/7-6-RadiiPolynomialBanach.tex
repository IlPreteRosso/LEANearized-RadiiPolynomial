% blueprint/src/Sections/Ch7/7-6-RadiiPolynomialBanach.tex

\section{Radii Polynomial Approach on Banach Spaces}\label{sec:radii_poly_banach}

This section extends the radii polynomial method to infinite-dimensional Banach spaces,
allowing for maps between potentially different spaces $E$ and $F$.

\subsection{Banach Space Setup}\label{subsec:banach_setup}

We work with two Banach spaces $E$ and $F$ over $\mathbb{R}$. For each space
$X \in \{E, F\}$:

  (1) \texttt{NormedAddCommGroup X}: $X$ has a norm satisfying definiteness,
        symmetry, triangle inequality
        
  (2) \texttt{NormedSpace $\mathbb{R}$ X}: The norm is compatible with scalar multiplication
  
  (3) \texttt{CompleteSpace X}: Every Cauchy sequence converges
        (crucial for fixed point theorems)


This framework supports:

  (4) Fréchet derivatives (via the norm structure)
  
  (5) Fixed point theorems (via completeness)
  
  (6) Mean Value Theorem (via the metric structure)
  
  (7) Linear operator theory (via the vector space structure)


\subsection{Neumann Series and Operator Invertibility}\label{subsec:neumann_series}

The Neumann series provides a constructive way to show operators close to the identity
are invertible.

\begin{theorem}[Neumann series invertibility]\label{thm:neumann_series}
  \lean{isUnit_of_norm_sub_id_lt_one}
  \leanok
  \uses{def:operator_norm}
  Let $E$ be a Banach space and $B: E \to E$ a continuous linear operator.
  If $\lVert I_E - B\rVert < 1$, then $B$ is invertible (a unit in the multiplicative sense).
\end{theorem}

\begin{proof}
  \leanok
  Write $B = I_E - (I_E - B)$. Since $\lVert I_E - B \rVert < 1$, the Neumann series
  $\sum_{n=0}^{\infty} (I_E - B)^n$ converges absolutely to $(I_E - (I_E - B))^{-1} = B^{-1}$.
  This is a direct application of Mathlib's \texttt{isUnit\_one\_sub\_of\_norm\_lt\_one}.
\end{proof}

\begin{lemma}[Explicit two-sided inverse]\label{lem:explicit_inverse}
  \lean{invertible_of_norm_sub_id_lt_one}
  \leanok
  \uses{thm:neumann_series}
  If $\lVert I_E - B\rVert < 1$ for $B: E \to E$, then there exists $B^{-1}: E \to E$ such that
  \begin{equation*}
  B \circ B^{-1} = I_E \quad \text{and} \quad B^{-1} \circ B = I_E.
  \end{equation*}
\end{lemma}

\begin{lemma}[Composition form]\label{lem:invertible_comp_form}
  \lean{invertible_comp_form}
  \leanok
  \uses{lem:explicit_inverse}
  If $\lVert I_E - B\rVert < 1$, then there exists $B^{-1}$ such that
  \begin{equation*}
  B.\text{comp}(B^{-1}) = I_E \quad \text{and} \quad B^{-1}.\text{comp}(B) = I_E.
  \end{equation*}
\end{lemma}

\subsection{Newton-Like Operators for E to F Maps}\label{subsec:newton_ef}

\begin{definition}[Newton-like map for E to F]\label{def:newton_ef}
  \lean{NewtonLikeMap}
  \leanok
  \uses{def:newton_like_map}
  For a function $f: E \to F$ and an approximate inverse $A: F \to E$, the
  \textit{Newton-like map} is
  \begin{equation*}
  T(x) = x - A(f(x)).
  \end{equation*}
  Note that $T: E \to E$ even though $f$ maps between different spaces.
\end{definition}

\begin{proposition}[Fixed points $\Leftrightarrow$ Zeros for E to F]\label{prop:fixed_zero_ef}
  \lean{fixedPoint_injective_iff_zero}
  \leanok
  \uses{def:newton_ef,prop:fixed_point_iff_zero}
  Let $f: E \to F$ and $A: F \to E$ be injective. Then for the Newton-like operator
  $T(x) = x - A(f(x))$:
  \begin{equation*}
  T(x) = x \quad \iff \quad f(x) = 0.
  \end{equation*}
  This holds even when $E \neq F$; injectivity of $A$ is sufficient.
\end{proposition}

\subsection{Radii Polynomial Definitions}\label{subsec:radii_poly_defs_banach}

\begin{definition}[General radii polynomial]\label{def:general_radii_poly}
  \lean{generalRadiiPolynomial}
  \leanok
  For constants $Y_0, Z_0, Z_1 \geq 0$ and function $Z_2: (0,\infty) \to [0,\infty)$,
  the \textit{general radii polynomial} is
  \begin{equation*}
  p(r) := Z_2(r)r^2 - (1 - Z_0 - Z_1)r + Y_0.
  \end{equation*}
\end{definition}

\begin{definition}[Combined bound (general case)]\label{def:Z_bound_general}
  \lean{Z_bound_general}
  \leanok
  The \textit{combined bound} is
  \begin{equation*}
  Z(r) := Z_0 + Z_1 + Z_2(r) \cdot r.
  \end{equation*}
\end{definition}

\begin{lemma}[Alternative form (general)]\label{lem:general_poly_alt_form}
  \lean{generalRadiiPolynomial_alt_form}
  \leanok
  \uses{def:general_radii_poly,def:Z_bound_general}
  The general radii polynomial can be rewritten as $p(r) = (Z(r) - 1)r + Y_0$.
\end{lemma}

\begin{lemma}[General polynomial negativity implies contraction]\label{lem:general_poly_neg_implies_contraction}
  \lean{general_radii_poly_neg_implies_Z_lt_one}
  \leanok
  \uses{def:general_radii_poly,def:Z_bound_general,lem:general_poly_alt_form}
  If $Y_0 \geq 0$, $r_0 > 0$, and $p(r_0) < 0$ for the general radii polynomial, then $Z(r_0) < 1$.
\end{lemma}

\begin{definition}[Simple radii polynomial]\label{def:simple_radii_poly_banach}
  \lean{simpleRadiiPolynomial}
  \leanok
  For $Y_0 \geq 0$ and $Z: (0,\infty) \to [0,\infty)$, the \textit{simple radii polynomial} is
  \begin{equation*}
  p(r) := (Z(r) - 1)r + Y_0.
  \end{equation*}
\end{definition}

\begin{lemma}[Simple polynomial negativity implies contraction]\label{lem:simple_poly_neg_implies_contraction}
  \lean{simple_radii_poly_neg_implies_Z_lt_one}
  \leanok
  \uses{def:simple_radii_poly_banach}
  If $Y_0 \geq 0$, $r_0 > 0$, and $p(r_0) < 0$ for the simple radii polynomial, then $Z(r_0) < 1$.
\end{lemma}

\subsection{Operator Bounds}\label{subsec:operator_bounds}

\begin{lemma}[$Y_0$ bound for Newton operator]\label{lem:newton_Y_bound}
  \lean{newton_operator_Y_bound}
  \leanok
  \uses{def:newton_ef}
  If $\lVert A(f(\bar{x}))\rVert \leq Y_0$, then for $T(x) = x - A(f(x))$:
  $\lVert T(\bar{x}) - \bar{x}\rVert \leq Y_0$.
\end{lemma}

\begin{proof}
  \leanok
  $\lVert T(\bar{x}) - \bar{x}\rVert = \lVert (\bar{x} - A(f(\bar{x}))) - \bar{x}\rVert = \lVert -A(f(\bar{x}))\rVert = \lVert A(f(\bar{x}))\rVert \leq Y_0$.
\end{proof}

\begin{lemma}[Derivative of Newton operator]\label{lem:newton_fderiv}
  \lean{newton_operator_fderiv}
  \leanok
  \uses{def:newton_ef}
  For $T(x) = x - A(f(x))$ where $f: E \to F$ is differentiable:
  $\text{DT}(x) = I_E - A \circ \text{Df}(x)$.
\end{lemma}

\begin{proof}
  \leanok
  By the chain rule: $\text{D}[x \mapsto A(f(x))] = A \circ \text{Df}(x)$.
  Since $\text{D}[\text{id}] = I_E$, we have $\text{DT}(x) = I_E - A \circ \text{Df}(x)$.
\end{proof}

\begin{lemma}[General derivative bound]\label{lem:newton_deriv_bound_general}
  \lean{newton_operator_derivative_bound_general}
  \leanok
  \uses{lem:newton_fderiv,def:Z_bound_general}
  Suppose for all $c \in \overline{B}_r(\bar{x})$:
  \begin{equation*}
  \lVert I_E - A \circ A^\dagger\rVert \leq Z_0
  \end{equation*}
  \begin{equation*}
  \lVert A \circ (A^\dagger - \text{Df}(\bar{x}))\rVert \leq Z_1
  \end{equation*}
  \begin{equation*}
  \lVert A \circ (\text{Df}(c) - \text{Df}(\bar{x}))\rVert \leq Z_2(r) \cdot r
  \end{equation*}
  Then $\lVert \text{DT}(c)\rVert \leq Z_0 + Z_1 + Z_2(r) \cdot r = Z(r)$.
\end{lemma}

\begin{proof}
  \leanok
  Decompose using $A^\dagger$:
  \begin{equation*}
  I_E - A \circ \text{Df}(c) = [I_E - A \circ A^\dagger] + A \circ [A^\dagger - \text{Df}(\bar{x})]
       + A \circ [\text{Df}(\bar{x}) - \text{Df}(c)].
  \end{equation*}
  By triangle inequality: $\lVert \text{DT}(c)\rVert \leq Z_0 + Z_1 + Z_2(r) \cdot r$.
\end{proof}

\begin{lemma}[Simple derivative bound]\label{lem:newton_deriv_bound_simple}
  \lean{newton_operator_derivative_bound_simple}
  \leanok
  \uses{lem:newton_fderiv,def:Z_bound}
  When $A^\dagger = \text{Df}(\bar{x})$ (so $Z_1 = 0$), for all $c \in \overline{B}_r(\bar{x})$:
  if $\lVert I_E - A \circ \text{Df}(\bar{x})\rVert \leq Z_0$ and
  $\lVert A \circ (\text{Df}(c) - \text{Df}(\bar{x}))\rVert \leq Z_2(r) \cdot r$,
  then $\lVert \text{DT}(c)\rVert \leq Z_0 + Z_2(r) \cdot r$.
\end{lemma}

\subsection{Helper Lemmas}\label{subsec:helper_lemmas}

\begin{lemma}[Closed balls are complete]\label{lem:closed_ball_complete}
  \lean{isComplete_closedBall}
  \leanok
  \uses{def:complete_metric_space}
  In a complete space $E$, closed balls $\overline{B}_r(x)$ are complete.
\end{lemma}

\begin{lemma}[Extended distance is finite]\label{lem:edist_finite}
  \lean{edist_ne_top_of_normed}
  \leanok
  In normed spaces, extended distance is always finite: $d_{\text{ext}}(x,y) \neq \top$.
\end{lemma}

\begin{lemma}[Constructing derivative inverse]\label{lem:construct_deriv_inverse}
  \lean{construct_derivative_inverse}
  \leanok
  \uses{lem:invertible_comp_form}
  If $A: F \to E$ is injective and $\lVert I_E - A \circ B\rVert < 1$ for $B: E \to F$,
  then $B$ is invertible with inverse $B^{-1} = (A \circ B)^{-1} \circ A$.
\end{lemma}

\begin{proof}
  \leanok
  By Lemma~\ref{lem:invertible_comp_form}, $A \circ B$ is invertible with inverse $(A \circ B)^{-1}$.
  Let $B^{-1} = (A \circ B)^{-1} \circ A$.
  
  \textbf{Left inverse:} $B(B^{-1}(x)) = B((A \circ B)^{-1}(A(x)))$.
  Apply $A$: $A(B(B^{-1}(x))) = (A \circ B)((A \circ B)^{-1}(A(x))) = A(x)$.
  By injectivity of $A$: $B(B^{-1}(x)) = x$.
  
  \textbf{Right inverse:} $B^{-1}(B(x)) = (A \circ B)^{-1}(A(B(x))) = (A \circ B)^{-1}((A \circ B)(x)) = x$.
\end{proof}

\begin{lemma}[Ball mapping property]\label{lem:maps_ball_to_itself}
  \lean{simple_maps_closedBall_to_itself}
  \leanok
  \uses{cor:mean_value_inequality,def:simple_radii_poly_banach}
  Given $T: E \to E$ differentiable with:
  (a) $\lVert T(\bar{x}) - \bar{x}\rVert \leq Y_0$;
  (b) $\lVert \text{DT}(c)\rVert \leq Z(r_0)$ for all $c \in \overline{B}_{r_0}(\bar{x})$;
  (c) $Z(r_0) \geq 0$;
  (d) $p(r_0) = (Z(r_0) - 1)r_0 + Y_0 < 0$.
  Then $T: \overline{B}_{r_0}(\bar{x}) \to \overline{B}_{r_0}(\bar{x})$.
\end{lemma}

\begin{proof}
  \leanok
  From $p(r_0) < 0$: $Z(r_0) \cdot r_0 + Y_0 < r_0$.
  The segment $[\bar{x}, x]$ lies in $\overline{B}_{r_0}(\bar{x})$ by convexity.
  By Mean Value Theorem: $\lVert T(x) - T(\bar{x})\rVert \leq Z(r_0) \cdot \lVert x - \bar{x}\rVert \leq Z(r_0) \cdot r_0$.
  By triangle inequality:
  \begin{equation*}
  \lVert T(x) - \bar{x}\rVert \leq \lVert T(x) - T(\bar{x})\rVert + \lVert T(\bar{x}) - \bar{x}\rVert
  \leq Z(r_0) \cdot r_0 + Y_0 < r_0.
  \end{equation*}
\end{proof}

\subsection{Main Theorems}\label{subsec:main_theorems}

\begin{theorem}[General Fixed Point Theorem (Theorem 7.6.1)]\label{thm:general_fixed_point}
  \lean{general_fixed_point_theorem}
  \leanok
  \uses{def:simple_radii_poly_banach,lem:maps_ball_to_itself,thm:contraction_mapping,lem:closed_ball_complete,lem:edist_finite,lem:simple_poly_neg_implies_contraction}
  Let $T: E \to E$ be Fréchet differentiable and $\bar{x} \in E$. Suppose:
  \begin{equation*}
  \lVert T(\bar{x}) - \bar{x}\rVert \leq Y_0
  \end{equation*}
  \begin{equation*}
  \lVert \text{DT}(x)\rVert \leq Z(r) \quad \text{for all } x \in \overline{B}_r(\bar{x})
  \end{equation*}
  Define $p(r) := (Z(r) - 1)r + Y_0$.

  If there exists $r_0 > 0$ such that $p(r_0) < 0$, then there exists a unique
  $\tilde{x} \in \overline{B}_{r_0}(\bar{x})$ such that $T(\tilde{x}) = \tilde{x}$.
\end{theorem}

\begin{proof}
  \leanok
  \textbf{Step 1:} From $p(r_0) < 0$ and Lemma~\ref{lem:simple_poly_neg_implies_contraction}: $Z(r_0) < 1$.
  Also $Z(r_0) \geq 0$ since $\lVert \text{DT}(\bar{x})\rVert \geq 0$.

  \textbf{Step 2:} By Lemma~\ref{lem:maps_ball_to_itself}, $T: \overline{B}_{r_0}(\bar{x}) \to \overline{B}_{r_0}(\bar{x})$.

  \textbf{Step 3:} $T$ restricted to $\overline{B}_{r_0}(\bar{x})$ is a contraction with constant $Z(r_0) < 1$.
  For $x, y \in \overline{B}_{r_0}(\bar{x})$, by Mean Value Theorem:
  $\lVert T(x) - T(y)\rVert \leq Z(r_0) \lVert x - y\rVert$.

  \textbf{Step 4:} The closed ball is complete by Lemma~\ref{lem:closed_ball_complete}.

  \textbf{Step 5:} Apply Theorem~\ref{thm:contraction_mapping} to get unique fixed point.
\end{proof}

\begin{theorem}[General Radii Polynomial Theorem (Theorem 7.6.2)]\label{thm:general_radii_poly_theorem}
  \lean{general_radii_polynomial_theorem}
  \leanok
  \uses{thm:general_fixed_point,prop:fixed_zero_ef,def:general_radii_poly,lem:newton_Y_bound,lem:newton_deriv_bound_general,lem:general_poly_alt_form,lem:general_poly_neg_implies_contraction}
  Let $E$ and $F$ be Banach spaces and $f: E \to F$ be Fréchet differentiable.
  Suppose $\bar{x} \in E$, $A^\dagger: E \to F$, and $A: F \to E$ with $A$ injective.
  Assume:
  \begin{equation*}
  \lVert A(f(\bar{x}))\rVert \leq Y_0
  \end{equation*}
  \begin{equation*}
  \lVert I_E - A \circ A^\dagger\rVert \leq Z_0
  \end{equation*}
  \begin{equation*}
  \lVert A \circ [\text{Df}(\bar{x}) - A^\dagger]\rVert \leq Z_1
  \end{equation*}
  \begin{equation*}
  \lVert A \circ [\text{Df}(c) - \text{Df}(\bar{x})]\rVert \leq Z_2(r) \cdot r
    \quad \text{for } c \in \overline{B}_r(\bar{x})
  \end{equation*}
  Define $p(r) := Z_2(r)r^2 - (1 - Z_0 - Z_1)r + Y_0$.

  If there exists $r_0 > 0$ such that $p(r_0) < 0$, then there exists a unique
  $\tilde{x} \in \overline{B}_{r_0}(\bar{x})$ with $f(\tilde{x}) = 0$.
\end{theorem}

\begin{proof}
  \leanok
  \textbf{Step 1:} Let $T(x) = x - A(f(x))$. Then $T: E \to E$ is differentiable.

  \textbf{Step 2:} By Lemma~\ref{lem:newton_Y_bound}: $\lVert T(\bar{x}) - \bar{x}\rVert \leq Y_0$.
  By Lemma~\ref{lem:newton_deriv_bound_general}: $\lVert \text{DT}(c)\rVert \leq Z(r_0)$ for $c \in \overline{B}_{r_0}(\bar{x})$.
  By Lemma~\ref{lem:general_poly_alt_form}: $p(r_0) = (Z(r_0) - 1)r_0 + Y_0 < 0$.

  \textbf{Step 3:} Apply Theorem~\ref{thm:general_fixed_point}: unique $\tilde{x} \in \overline{B}_{r_0}(\bar{x})$ with $T(\tilde{x}) = \tilde{x}$.

  \textbf{Step 4:} By Proposition~\ref{prop:fixed_zero_ef} with injectivity of $A$: $f(\tilde{x}) = 0$.
\end{proof}

\begin{theorem}[Simple Radii Polynomial for E to F]\label{thm:simple_radii_poly_ef}
  \lean{simple_radii_polynomial_theorem_EtoF}
  \leanok
  \uses{thm:general_fixed_point,prop:fixed_zero_ef,def:radii_polynomial,lem:newton_Y_bound,lem:newton_deriv_bound_simple,lem:radii_poly_alt_form}
  Given $f: E \to F$ Fréchet differentiable and injective $A: F \to E$ satisfying
  $\lVert A(f(\bar{x}))\rVert \leq Y_0$,
  $\lVert I_E - A \circ \text{Df}(\bar{x})\rVert \leq Z_0$, and
  $\lVert A \circ [\text{Df}(c) - \text{Df}(\bar{x})]\rVert \leq Z_2(r) \cdot r$ for $c \in \overline{B}_r(\bar{x})$.
  If $p(r_0) = Z_2(r_0)r_0^2 - (1-Z_0)r_0 + Y_0 < 0$, then there exists unique
  $\tilde{x} \in \overline{B}_{r_0}(\bar{x})$ with $f(\tilde{x}) = 0$.
\end{theorem}

\begin{theorem}[Simple Radii Polynomial (Same Space)]\label{thm:simple_radii_poly_same}
  \lean{simple_radii_polynomial_theorem_same_space}
  \leanok
  \uses{thm:general_fixed_point,prop:fixed_zero_ef,lem:construct_deriv_inverse,def:radii_polynomial,lem:newton_Y_bound,lem:newton_deriv_bound_simple,lem:poly_neg_implies_contraction}
  Consider $f: E \to E$ Fréchet differentiable, $\bar{x} \in E$, and $A: E \to E$
  injective. Assume
  $\lVert A(f(\bar{x}))\rVert \leq Y_0$,
  $\lVert I_E - A \circ \text{Df}(\bar{x})\rVert \leq Z_0$, and
  $\lVert A \circ [\text{Df}(c) - \text{Df}(\bar{x})]\rVert \leq Z_2(r) \cdot r$ for $c \in \overline{B}_r(\bar{x})$.
  Define $p(r) := Z_2(r)r^2 - (1-Z_0)r + Y_0$.

  If $p(r_0) < 0$, then there exists unique $\tilde{x} \in \overline{B}_{r_0}(\bar{x})$
  with $f(\tilde{x}) = 0$ and $\text{Df}(\tilde{x})$ invertible.
\end{theorem}

\begin{proof}
  \leanok
  \textbf{Steps 1--4:} As in Theorem~\ref{thm:general_radii_poly_theorem}, get fixed point $\tilde{x}$ with $f(\tilde{x}) = 0$.

  \textbf{Step 5 (Invertibility):} Since $\tilde{x} \in \overline{B}_{r_0}(\bar{x})$ and $Z(r_0) < 1$
  (by Lemma~\ref{lem:poly_neg_implies_contraction}),
  $\lVert I_E - A \circ \text{Df}(\tilde{x})\rVert \leq Z(r_0) < 1$.
  Apply Lemma~\ref{lem:construct_deriv_inverse}: $\text{Df}(\tilde{x})$ is invertible.
\end{proof}