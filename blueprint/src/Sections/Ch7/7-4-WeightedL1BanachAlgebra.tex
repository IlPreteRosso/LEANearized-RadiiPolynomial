% blueprint/src/Sections/Ch7/7-4-WeightedL1BanachAlgebra.tex

\section{Weighted \texorpdfstring{$\ell^1_\nu$}{ℓ¹ν} Banach Algebra (Section 7.4)}\label{sec:weighted_l1}

This section establishes the weighted $\ell^1_\nu$ space as a commutative Banach algebra
under the Cauchy product. The algebraic structure is derived by connecting to
\texttt{PowerSeries} multiplication, then the analytic properties (submultiplicativity)
are proven using weight factorization.

\subsection{Positive Reals and Scaled Spaces}\label{subsec:scaled_real}

\begin{definition}[Positive real numbers]\label{def:posreal}
  \lean{PosReal}
  \leanok
  The type of positive real numbers:
  $\texttt{PosReal} := \{x : \mathbb{R} \mid 0 < x\}$.
\end{definition}

\begin{definition}[Scaled real type]\label{def:ScaledReal}
  \lean{ScaledReal}
  \leanok
  \uses{def:posreal}
  For $\nu > 0$ and $n \in \mathbb{N}$, the type $\texttt{ScaledReal}\ \nu\ n$ is $\mathbb{R}$
  equipped with the scaled norm $\|x\|_n := |x| \cdot \nu^n$.
\end{definition}

\subsection{Weighted \texorpdfstring{$\ell^p$}{ℓᵖ} Spaces}\label{subsec:weighted_lp}

\begin{definition}[Weighted $\ell^p$ space]\label{def:lpWeighted}
  \lean{lpWeighted}
  \leanok
  \uses{def:ScaledReal}
  The weighted $\ell^p_\nu$ space is realized as $\texttt{lp}\ (\texttt{ScaledReal}\ \nu)\ p$.
  The norm is:
  \begin{equation*}
  \|a\|_{p,\nu} := \left(\sum_{n=0}^\infty |a_n|^p \nu^{pn}\right)^{1/p}.
  \end{equation*}
\end{definition}

\begin{definition}[Weighted $\ell^1$ space]\label{def:l1Weighted}
  \lean{l1Weighted}
  \leanok
  \uses{def:lpWeighted,def:complete_metric_space}
  Specialization of $\ell^p_\nu$ to $p = 1$:
  \begin{equation*}
  \ell^1_\nu = \left\{ a : \mathbb{N} \to \mathbb{R} \;\middle|\; \|a\|_{1,\nu} := \sum_{n=0}^\infty |a_n| \nu^n < \infty \right\}.
  \end{equation*}
\end{definition}

\begin{lemma}[Completeness]\label{lem:lpWeighted_complete}
  \lean{lpWeighted.instCompleteSpace}
  \leanok
  \uses{def:lpWeighted}
  The space $\ell^p_\nu$ is complete (a Banach space) for $p \geq 1$.
\end{lemma}

\begin{lemma}[Membership criterion]\label{lem:l1_mem_iff}
  \lean{l1Weighted.mem_iff}
  \leanok
  \uses{def:l1Weighted}
  A sequence $a$ belongs to $\ell^1_\nu$ iff $\sum_{n=0}^\infty |a_n| \nu^n$ converges.
\end{lemma}

\begin{lemma}[Weight factorization]\label{lem:antidiagonal_weight}
  \lean{l1Weighted.antidiagonal_weight}
  \leanok
  \uses{def:posreal}
  For $k + l = n$: $\nu^k \cdot \nu^l = \nu^n$.
  This is the key property enabling submultiplicativity.
\end{lemma}


\subsection{Cauchy Product (Definition 7.4.2)}\label{subsec:cauchy_product}

The Cauchy product (convolution) defines multiplication on sequence spaces.

\begin{definition}[Cauchy product]\label{def:CauchyProduct}
  \lean{CauchyProduct}
  \leanok
  The \emph{Cauchy product} of sequences $a, b : \mathbb{N} \to R$ is:
  \begin{equation*}
  (a \star b)_n = \sum_{k+l=n} a_k b_l = \sum_{j=0}^{n} a_{n-j} b_j.
  \end{equation*}
\end{definition}

\begin{theorem}[Connection to PowerSeries]\label{thm:toPowerSeries_mul}
  \lean{CauchyProduct.toPowerSeries_mul}
  \leanok
  \uses{def:CauchyProduct}
  Let $\texttt{toPowerSeries}(a) = \sum_n a_n X^n$. Then:
  \begin{equation*}
  \texttt{toPowerSeries}(a \star b) = \texttt{toPowerSeries}(a) \cdot \texttt{toPowerSeries}(b).
  \end{equation*}
  This bridge lets us transport all ring axioms from \texttt{PowerSeries R}.
\end{theorem}

\begin{theorem}[Associativity]\label{thm:CauchyProduct_assoc}
  \lean{CauchyProduct.assoc}
  \leanok
  \uses{thm:toPowerSeries_mul}
  $(a \star b) \star c = a \star (b \star c)$.
\end{theorem}

\begin{theorem}[Commutativity]\label{thm:CauchyProduct_comm}
  \lean{CauchyProduct.comm}
  \leanok
  \uses{thm:toPowerSeries_mul}
  $a \star b = b \star a$ (when $R$ is commutative).
\end{theorem}

\begin{theorem}[Left distributivity]\label{thm:CauchyProduct_left_distrib}
  \lean{CauchyProduct.left_distrib}
  \leanok
  \uses{def:CauchyProduct}
  $a \star (b + c) = a \star b + a \star c$.
\end{theorem}

\begin{theorem}[Right distributivity]\label{thm:CauchyProduct_right_distrib}
  \lean{CauchyProduct.right_distrib}
  \leanok
  \uses{def:CauchyProduct}
  $(a + b) \star c = a \star c + b \star c$.
\end{theorem}

\begin{definition}[Identity sequence]\label{def:CauchyProduct_one}
  \lean{CauchyProduct.one}
  \leanok
  The identity element is the Kronecker delta:
  \begin{equation*}
  e_n = \delta_{n,0} = \begin{cases} 1 & n = 0 \\ 0 & n \geq 1. \end{cases}
  \end{equation*}
\end{definition}

\begin{theorem}[Left identity]\label{thm:CauchyProduct_one_mul}
  \lean{CauchyProduct.one_mul}
  \leanok
  \uses{def:CauchyProduct_one,thm:toPowerSeries_mul}
  $e \star a = a$.
\end{theorem}

\begin{theorem}[Right identity]\label{thm:CauchyProduct_mul_one}
  \lean{CauchyProduct.mul_one}
  \leanok
  \uses{def:CauchyProduct_one,thm:toPowerSeries_mul}
  $a \star e = a$.
\end{theorem}


\subsection{Scalar-Sequence Compatibility}\label{subsec:scalar_compat}

These lemmas establish compatibility between scalar multiplication and the Cauchy product.
They are essential for the Fr\'echet derivative formula in Section~\ref{sec:example_7_7}.

\begin{theorem}[Left scalar multiplication]\label{thm:CauchyProduct_smul_mul}
  \lean{CauchyProduct.smul_mul}
  \leanok
  \uses{def:CauchyProduct}
  Scalars pull out on the left: $(c \cdot a) \star b = c \cdot (a \star b)$.
\end{theorem}

\begin{theorem}[Right scalar multiplication]\label{thm:CauchyProduct_mul_smul}
  \lean{CauchyProduct.mul_smul}
  \leanok
  \uses{def:CauchyProduct,thm:CauchyProduct_comm}
  Scalars pull out on the right: $a \star (c \cdot b) = c \cdot (a \star b)$.
  Requires commutativity of the coefficient ring.
\end{theorem}


\subsection{Banach Algebra Structure (Theorem 7.4.4)}\label{subsec:banach_algebra}

\begin{lemma}[Closure under multiplication]\label{lem:l1Weighted_mem}
  \lean{l1Weighted.mem}
  \leanok
  \uses{def:l1Weighted,def:CauchyProduct,lem:antidiagonal_weight}
  If $a, b \in \ell^1_\nu$, then $a \star b \in \ell^1_\nu$.
\end{lemma}

\begin{proof}
  \leanok
  By Mertens' theorem and weight factorization:
  \begin{align*}
  \sum_{n=0}^\infty |(a \star b)_n| \nu^n 
  &\leq \sum_{n=0}^\infty \sum_{k=0}^n |a_k| |b_{n-k}| \nu^n \\
  &= \sum_{n=0}^\infty \sum_{k=0}^n |a_k| \nu^k \cdot |b_{n-k}| \nu^{n-k} \\
  &= \|a\|_{\ell^1_\nu} \cdot \|b\|_{\ell^1_\nu} < \infty.
  \end{align*}
\end{proof}

\begin{definition}[Multiplication on $\ell^1_\nu$]\label{def:l1Weighted_mul}
  \lean{l1Weighted.mul}
  \leanok
  \uses{def:l1Weighted,def:CauchyProduct,lem:l1Weighted_mem}
  $\texttt{mul}(a, b) := a \star b$ lifted to $\ell^1_\nu$.
\end{definition}

\begin{theorem}[Submultiplicativity]\label{thm:norm_mul_le}
  \lean{l1Weighted.norm_mul_le}
  \leanok
  \uses{def:l1Weighted_mul,lem:antidiagonal_weight}
  $\|a \star b\|_{1,\nu} \leq \|a\|_{1,\nu} \cdot \|b\|_{1,\nu}$.
  
  This is \textbf{axiom (4)} of Definition 7.4.1, the key analytic property.
  The proof uses Mertens' theorem and weight factorization $\nu^n = \nu^k \cdot \nu^l$.
\end{theorem}

\begin{definition}[Identity element]\label{def:l1Weighted_one}
  \lean{l1Weighted.one}
  \leanok
  \uses{def:l1Weighted,def:CauchyProduct_one}
  The identity $e \in \ell^1_\nu$ with $e_0 = 1$, $e_n = 0$ for $n \geq 1$.
\end{definition}

\begin{lemma}[Norm of identity]\label{lem:norm_one}
  \lean{l1Weighted.norm_one}
  \leanok
  \uses{def:l1Weighted_one}
  $\|e\|_{1,\nu} = 1$.
\end{lemma}


\subsection{Typeclass Instances (Corollary 7.4.5)}\label{subsec:l1_instances}

\begin{theorem}[Ring instance]\label{thm:l1Weighted_Ring}
  \lean{l1Weighted.instRing}
  \leanok
  \uses{def:l1Weighted_mul,def:l1Weighted_one,thm:CauchyProduct_assoc,thm:CauchyProduct_left_distrib,thm:CauchyProduct_right_distrib,thm:CauchyProduct_one_mul,thm:CauchyProduct_mul_one}
  $\ell^1_\nu$ is a ring under Cauchy product multiplication.
\end{theorem}

\begin{theorem}[Commutative ring instance]\label{thm:l1Weighted_CommRing}
  \lean{l1Weighted.instCommRing}
  \leanok
  \uses{thm:l1Weighted_Ring,thm:CauchyProduct_comm}
  $\ell^1_\nu$ is a commutative ring.
\end{theorem}

\begin{theorem}[Normed ring instance]\label{thm:l1Weighted_NormedRing}
  \lean{l1Weighted.instNormedRing}
  \leanok
  \uses{thm:l1Weighted_Ring,thm:norm_mul_le}
  $\ell^1_\nu$ is a normed ring (submultiplicativity holds).
\end{theorem}

\begin{theorem}[Norm one class]\label{thm:l1Weighted_NormOneClass}
  \lean{l1Weighted.instNormOneClass}
  \leanok
  \uses{thm:l1Weighted_NormedRing,lem:norm_one}
  $\|1\| = 1$ in $\ell^1_\nu$.
\end{theorem}


\subsection{Algebra Structure}\label{subsec:algebra_structure}

These instances establish that $\ell^1_\nu$ is an $\mathbb{R}$-algebra, meaning scalar multiplication
is compatible with ring multiplication. This is essential for the Fr\'echet derivative formula
$D[\mathrm{sq}](a) h = 2(a \star h)$, where the scalar $2$ must commute with the multiplication.

\begin{theorem}[Scalar commutes with multiplication]\label{thm:l1Weighted_SMulCommClass}
  \lean{l1Weighted.instSMulCommClass}
  \leanok
  \uses{thm:l1Weighted_Ring,thm:CauchyProduct_mul_smul}
  For $c \in \mathbb{R}$ and $a, b \in \ell^1_\nu$:
  \begin{equation*}
  c \cdot (a \star b) = a \star (c \cdot b).
  \end{equation*}
  This is the \texttt{SMulCommClass} instance.
\end{theorem}

\begin{theorem}[Scalar tower property]\label{thm:l1Weighted_IsScalarTower}
  \lean{l1Weighted.instIsScalarTower}
  \leanok
  \uses{thm:l1Weighted_Ring,thm:CauchyProduct_smul_mul}
  For $c \in \mathbb{R}$ and $a, b \in \ell^1_\nu$:
  \begin{equation*}
  (c \cdot a) \star b = c \cdot (a \star b).
  \end{equation*}
  This is the \texttt{IsScalarTower} instance.
\end{theorem}


\begin{theorem}[$\mathbb{R}$-Algebra instance]\label{thm:l1Weighted_Algebra}
  \lean{l1Weighted.instAlgebra}
  \leanok
  \uses{thm:l1Weighted_Ring,thm:l1Weighted_SMulCommClass,thm:l1Weighted_IsScalarTower}
  $\ell^1_\nu$ is an $\mathbb{R}$-algebra. The algebra map $\mathbb{R} \to \ell^1_\nu$
  sends $r$ to $r \cdot e$ where $e$ is the identity sequence.
\end{theorem}

\begin{lemma}[Norm of scalar multiplication]\label{lem:norm_smul}
  % Uses Mathlib's norm_smul for NormedSpace
  \uses{def:l1Weighted}
  For $c \in \mathbb{R}$ and $a \in \ell^1_\nu$:
  \begin{equation*}
  \|c \cdot a\|_{1,\nu} = |c| \cdot \|a\|_{1,\nu}.
  \end{equation*}
\end{lemma}

\begin{proof}
  \leanok
  Direct computation:
  $\|c \cdot a\| = \sum_n |c \cdot a_n| \nu^n = |c| \sum_n |a_n| \nu^n = |c| \cdot \|a\|$.
\end{proof}

\begin{theorem}[Normed $\mathbb{R}$-Algebra instance]\label{thm:l1Weighted_NormedAlgebra}
  \lean{l1Weighted.instNormedAlgebra}
  \leanok
  \uses{thm:l1Weighted_Algebra,thm:l1Weighted_NormedRing,lem:norm_smul}
  $\ell^1_\nu$ is a normed $\mathbb{R}$-algebra, satisfying $\|c \cdot a\| \leq |c| \cdot \|a\|$.
  (In fact, equality holds.)
\end{theorem}

\begin{lemma}[Norm bound for powers]\label{lem:norm_pow_le}
  \lean{l1Weighted.norm_pow_le}
  \leanok
  \uses{thm:l1Weighted_NormedRing}
  For $a \in \ell^1_\nu$ and $n \in \mathbb{N}$:
  \begin{equation*}
  \|a^n\|_{1,\nu} \leq \|a\|_{1,\nu}^n.
  \end{equation*}
\end{lemma}

\begin{proof}
  \leanok
  By induction on $n$. The base case $n = 0$ gives $\|1\| = 1 = \|a\|^0$.
  For the inductive step, submultiplicativity gives
  $\|a^{n+1}\| = \|a \cdot a^n\| \leq \|a\| \cdot \|a^n\| \leq \|a\| \cdot \|a\|^n = \|a\|^{n+1}$.
\end{proof}

\begin{corollary}[Commutative Banach algebra]\label{cor:l1Weighted_Banach_algebra}
  \uses{lem:lpWeighted_complete,thm:l1Weighted_CommRing,thm:l1Weighted_NormedRing,thm:l1Weighted_NormOneClass,thm:l1Weighted_Algebra,thm:l1Weighted_NormedAlgebra}
  $\ell^1_\nu$ is a commutative Banach algebra over $\mathbb{R}$:
  \begin{enumerate}
    \item Complete normed space (Banach)
    \item Commutative ring under $\star$
    \item $\mathbb{R}$-algebra with compatible scalar multiplication
    \item Submultiplicative: $\|a \star b\| \leq \|a\| \cdot \|b\|$
    \item Scalar-norm compatibility: $\|c \cdot a\| = |c| \cdot \|a\|$
    \item $\|1\| = 1$
  \end{enumerate}
  This is the full Mathlib \texttt{NormedAlgebra} + \texttt{CompleteSpace} structure.
\end{corollary}


\subsection{Architecture Summary}\label{subsec:architecture}

The formalization separates concerns into layers:

\begin{center}
\begin{tabular}{|l|l|l|}
\hline
\textbf{File} & \textbf{Contents} & \textbf{Dependencies} \\
\hline
\texttt{CauchyProduct.lean} & Ring axioms, scalar compat & \texttt{PowerSeries.Basic} \\
\texttt{lpWeighted.lean} & Banach structure, instances & \texttt{CauchyProduct}, \texttt{lpSpace} \\
\texttt{FrechetCauchyProduct.lean} & Fr\'echet derivatives & Ring instances \\
\hline
\end{tabular}
\end{center}

Key insight: Ring axioms are transported from \texttt{PowerSeries} via \texttt{toPowerSeries\_mul},
avoiding manual verification of associativity/distributivity. The \texttt{SMulCommClass} and 
\texttt{IsScalarTower} instances similarly lift scalar compatibility from \texttt{CauchyProduct}.