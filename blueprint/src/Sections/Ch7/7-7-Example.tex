% blueprint/src/Sections/Ch7/7-7-Example.tex

\section{Example 7.7: Square Root via Power Series}\label{sec:example_7_7}

This section applies the radii polynomial method to prove existence of an analytic 
branch of $\sqrt{\lambda}$ near a base point $\lambda_0 > 0$. We work in the 
weighted sequence space $\ell^1_\nu$ (Section~\ref{sec:weighted_l1}) and verify all 
bounds to establish existence of a fixed point, then interpret this fixed point as 
a convergent power series.

\subsection{The Fixed-Point Problem}\label{subsec:fixed_point_problem}

We seek an analytic function $x(\lambda) = \sum_{n=0}^\infty a_n (\lambda - \lambda_0)^n$ 
satisfying $x(\lambda)^2 = \lambda$. In coefficient space, this becomes: find 
$a \in \ell^1_\nu$ such that $(a \star a)_n = c_n$ where $c$ is the parameter sequence.

\begin{definition}[Parameter sequence]\label{def:param_seq}
  \lean{Example_7_7.paramSeq}
  \leanok
  \uses{def:l1Weighted}
  For $\lambda_0 \in \mathbb{R}$, the \textit{parameter sequence} is
  \begin{equation*}
  c_n = \begin{cases} \lambda_0 & n = 0 \\ 1 & n = 1 \\ 0 & n \geq 2 \end{cases}
  \end{equation*}
  This encodes the polynomial $\lambda_0 + z$ (i.e., $\lambda$ when $z = \lambda - \lambda_0$).
\end{definition}

\begin{definition}[The map $F$]\label{def:F_map}
  \lean{Example_7_7.F}
  \leanok
  \uses{def:l1Weighted,def:CauchyProduct,def:param_seq}
  The fixed-point map $F: \ell^1_\nu \to \ell^1_\nu$ is defined by
  \begin{equation*}
  F(a)_n = (a \star a)_n - c_n.
  \end{equation*}
  A zero of $F$ corresponds to a power series squaring to $\lambda_0 + z$.
\end{definition}

\begin{definition}[Squaring map]\label{def:sq_map}
  \lean{l1Weighted.sq}
  \leanok
  \uses{def:l1Weighted,def:CauchyProduct}
  The squaring map $\text{sq}: \ell^1_\nu \to \ell^1_\nu$ is $\text{sq}(a) = a \star a$.
\end{definition}

\begin{theorem}[Fr\'echet derivative of squaring]\label{thm:fderiv_sq}
  \lean{l1Weighted.hasFDerivAt_sq}
  \leanok
  \uses{def:sq_map,thm:l1Weighted_CommRing,thm:norm_mul_le,thm:l1Weighted_SMulCommClass}
  The squaring map is Fr\'echet differentiable with derivative at $a$:
  \begin{equation*}
  D_a[\text{sq}](h) = 2(a \star h).
  \end{equation*}
\end{theorem}

\begin{proof}
  \leanok
  The key identity $(a+h)^2 - a^2 - 2(a \star h) = h^2$ follows from the \texttt{CommRing}
  instance via the \texttt{ring} tactic. The remainder estimate $\|h^2\| \leq \|h\|^2$
  uses submultiplicativity, giving the little-o condition. The \texttt{SMulCommClass}
  instance ensures that $2 \cdot (a \star h) = a \star (2 \cdot h)$.
\end{proof}

\subsection{Approximate Solution}\label{subsec:approx_solution}

\begin{definition}[Approximate solution structure]\label{def:approx_solution}
  \lean{Example_7_7.ApproxSolution}
  \leanok
  \uses{def:l1Weighted}
  An \textit{approximate solution} of order $N$ consists of:
  \begin{itemize}
    \item Finite coefficients $\bar{a}_0, \bar{a}_1, \ldots, \bar{a}_N \in \mathbb{R}$
    \item The extension $\bar{a} \in \ell^1_\nu$ defined by $\bar{a}_n = 0$ for $n > N$
  \end{itemize}
\end{definition}

\begin{definition}[Concrete approximate solution]\label{def:concrete_approx}
  \lean{ConcreteExample.sol}
  \leanok
  \uses{def:approx_solution}
  For $\lambda_0 = 1$ and $N = 2$, the approximate solution is:
  \begin{equation*}
  \bar{a}_0 = 1, \quad \bar{a}_1 = \frac{1}{2}, \quad \bar{a}_2 = -\frac{1}{8}.
  \end{equation*}
  These are the first three Taylor coefficients of $\sqrt{1 + z}$ at $z = 0$.
\end{definition}

\subsection{Bound Definitions}\label{subsec:bound_defs}

\begin{definition}[$Y_0$ bound]\label{def:Y0_bound}
  \lean{Example_7_7.Y₀_bound}
  \leanok
  \uses{def:F_map,def:approx_solution}
  The $Y_0$ bound measures the initial defect:
  \begin{equation*}
  Y_0 = \lVert A \cdot F(\bar{a})\rVert_{\ell^1_\nu}
  \end{equation*}
  where $A$ is the approximate inverse. For diagonal $A$ with entries $A_{nn} = 1/(2\bar{a}_0)$:
  \begin{equation*}
  Y_0 = \sum_{n=0}^N |A_{nn} \cdot F(\bar{a})_n| \nu^n + Y_0^{\text{tail}}
  \end{equation*}
  where $Y_0^{\text{tail}}$ accounts for the tail contribution.
\end{definition}

\begin{definition}[$Z_0$ bound]\label{def:Z0_bound}
  \lean{Example_7_7.Z₀_bound}
  \leanok
  \uses{def:approx_solution,thm:fderiv_sq}
  The $Z_0$ bound measures deviation from identity:
  \begin{equation*}
  Z_0 = \lVert I - A \circ D_{\bar{a}}F\rVert.
  \end{equation*}
  Since $D_{\bar{a}}F(h) = 2(\bar{a} \star h)$ and $A$ is diagonal with $A_{nn} = 1/(2\bar{a}_0)$:
  \begin{equation*}
  Z_0 = \sup_{n \geq 0} \left|1 - \frac{2(\bar{a} \star e_n)_n}{2\bar{a}_0}\right| \cdot \nu^n
  \end{equation*}
  where $e_n$ is the $n$-th standard basis sequence.
\end{definition}

\begin{definition}[$Z_1$ bound]\label{def:Z1_bound}
  \lean{Example_7_7.Z₁_bound}
  \leanok
  \uses{def:approx_solution,thm:fderiv_sq}
  The $Z_1$ bound accounts for nonlinearity:
  \begin{equation*}
  Z_1 = \frac{1}{|\bar{a}_0|} \sum_{k=1}^N |\bar{a}_k| \nu^k.
  \end{equation*}
\end{definition}

\begin{definition}[$Z_2$ bound]\label{def:Z2_bound}
  \lean{Example_7_7.Z₂_bound}
  \leanok
  \uses{thm:fderiv_sq}
  The $Z_2$ bound captures derivative variation:
  \begin{equation*}
  Z_2 = 2 \cdot \max\left(\|A\|, \frac{1}{2|\bar{a}_0|}\right).
  \end{equation*}
\end{definition}

\subsection{Radii Polynomial}\label{subsec:radii_polynomial}

\begin{definition}[Radii polynomial]\label{def:radii_poly}
  \lean{Example_7_7.radiiPoly_7_7}
  \leanok
  \uses{def:Y0_bound,def:Z0_bound,def:Z1_bound,def:Z2_bound}
  The radii polynomial is
  \begin{equation*}
  p(r) = Z_2 r^2 - (1 - Z_0 - Z_1) r + Y_0.
  \end{equation*}
\end{definition}

\begin{theorem}[Existence condition]\label{thm:existence_condition}
  % \lean{Example_7_7.existence_of_root}  % TODO: implement
  \uses{def:radii_poly}
  If there exists $r_0 > 0$ such that $p(r_0) < 0$, then there exists a unique 
  $\tilde{a} \in \ell^1_\nu$ with $\|\tilde{a} - \bar{a}\| \leq r_0$ satisfying $F(\tilde{a}) = 0$.
\end{theorem}

\subsection{Verification}\label{subsec:verification}

\begin{theorem}[Bounds verification]\label{thm:bounds_verified}
  % \lean{ConcreteExample.bounds_verified}  % TODO: implement concrete numerical verification
  \uses{def:concrete_approx,def:Y0_bound,def:Z0_bound,def:Z1_bound,def:Z2_bound}
  For $\lambda_0 = 1/3$, $\nu = 1/4$, $N = 2$, the bounds satisfy:
  \begin{itemize}
    \item $Y_0 < 0.001$
    \item $Z_0 < 0.5$
    \item $Z_1 < 0.3$
    \item $Z_2 < 4$
  \end{itemize}
  These imply $p(r_0) < 0$ for some $r_0 > 0$.
\end{theorem}

\begin{theorem}[Example 7.7 existence]\label{thm:example_7_7_existence}
  \lean{Example_7_7.example_7_7_main_theorem}
  \leanok
  \uses{thm:bounds_verified,thm:existence_condition,thm:contraction_mapping}
  For $\lambda_0 = 1/3$ and $\nu = 1/4$, there exists a unique $\tilde{a} \in \ell^1_\nu$ 
  near $\bar{a}$ satisfying $\tilde{a} \star \tilde{a} = c$.
\end{theorem}

\subsection{Power Series Interpretation}\label{subsec:power_series_interp}

\subsubsection{Formal Power Series Connection}

\begin{definition}[Embedding into PowerSeries]\label{def:to_power_series}
  \lean{l1Weighted.toPowerSeries}
  \leanok
  \uses{def:l1Weighted}
  The embedding $\ell^1_\nu \hookrightarrow \texttt{PowerSeries}\ \mathbb{R}$ sends 
  $a \mapsto \sum_{n=0}^\infty a_n X^n$.
\end{definition}

\begin{definition}[Parameter power series]\label{def:param_power_series}
  \lean{Example_7_7.c}
  \leanok
  \uses{def:param_seq}
  For $\lambda_0 \in \mathbb{R}$, the parameter element $c \in \ell^1_\nu$ represents $\lambda_0 + z$:
  \begin{equation*}
  c = (\lambda_0, 1, 0, 0, \ldots) \in \ell^1_\nu.
  \end{equation*}
\end{definition}

\begin{theorem}[Multiplication equals Cauchy product]\label{thm:coeff_mul_eq_cauchy}
  \lean{l1Weighted.coeff_mul_eq_cauchyProduct}
  \leanok
  \uses{def:to_power_series,def:CauchyProduct}
  For $a, b \in \ell^1_\nu$:
  \begin{equation*}
  \texttt{coeff}_n(\texttt{toPowerSeries}(a) \cdot \texttt{toPowerSeries}(b)) = (a \star b)_n.
  \end{equation*}
  This is essentially definitional: multiplication in $\texttt{PowerSeries}\ \mathbb{R}$ 
  \emph{is} the Cauchy product.
\end{theorem}

\begin{theorem}[Formal fixed-point equation]\label{thm:formal_sq_eq_param}
  \lean{l1Weighted.toPowerSeries_sq_eq_param}
  \leanok
  \uses{def:to_power_series,def:param_power_series,thm:coeff_mul_eq_cauchy}
  If $F(\tilde{a}) = 0$ (i.e., $(\tilde{a} \star \tilde{a})_n = c_n$ for all $n$), then at the formal level:
  \begin{equation*}
  \texttt{toPowerSeries}(\tilde{a})^2 = \texttt{paramPowerSeries}(\lambda_0).
  \end{equation*}
\end{theorem}

\subsubsection{Analytic Evaluation}

\begin{definition}[Analytic evaluation]\label{def:eval}
  \lean{l1Weighted.eval}
  \leanok
  \uses{def:l1Weighted}
  For $a \in \ell^1_\nu$ and $z \in \mathbb{R}$, the evaluation is
  \begin{equation*}
  \texttt{eval}(a, z) := \sum_{n=0}^\infty a_n z^n.
  \end{equation*}
\end{definition}

\begin{lemma}[Absolute convergence]\label{lem:summable_eval}
  \lean{l1Weighted.summable_eval}
  \leanok
  \uses{def:l1Weighted,def:eval}
  If $a \in \ell^1_\nu$ and $|z| \leq \nu$, then $\sum_{n=0}^\infty a_n z^n$ 
  converges absolutely.
\end{lemma}

\begin{proof}
  \leanok
  For $|z| \leq \nu$:
  $\sum_{n=0}^\infty |a_n z^n| \leq \sum_{n=0}^\infty |a_n| \nu^n = \lVert a\rVert_{\ell^1_\nu} < \infty$.
\end{proof}

\begin{theorem}[Mertens' theorem: evaluation commutes with multiplication]\label{thm:eval_mul}
  \lean{l1Weighted.eval_mul}
  \leanok
  \uses{def:CauchyProduct,lem:summable_eval}
  For $a, b \in \ell^1_\nu$ and $|z| \leq \nu$:
  \begin{equation*}
  \texttt{eval}(a, z) \cdot \texttt{eval}(b, z) = \sum_{n=0}^\infty (a \star b)_n z^n.
  \end{equation*}
\end{theorem}

\begin{proof}
  \leanok
  By absolute convergence, apply Mertens' theorem: 
  the product of absolutely convergent series equals the series of Cauchy products.
\end{proof}

\begin{theorem}[Squaring identity]\label{thm:eval_sq_eq}
  \lean{l1Weighted.eval_sq_eq}
  \leanok
  \uses{thm:example_7_7_existence,thm:eval_mul,def:param_seq}
  If $\tilde{a} \in \ell^1_\nu$ satisfies $F(\tilde{a}) = 0$, then for $|z| \leq \nu$:
  \begin{equation*}
  \texttt{eval}(\tilde{a}, z)^2 = \lambda_0 + z.
  \end{equation*}
\end{theorem}

\begin{proof}
  \leanok
  By Theorem~\ref{thm:eval_mul}:
  $\texttt{eval}(\tilde{a}, z)^2 = \sum_{n=0}^\infty (\tilde{a} \star \tilde{a})_n z^n 
  = \sum_{n=0}^\infty c_n z^n = \lambda_0 + z$.
\end{proof}

\begin{corollary}[Evaluation identity]\label{cor:analytic_is_sqrt}
  \lean{Example_7_7.analyticSolution_is_sqrt}
  \leanok
  \uses{thm:eval_sq_eq}
  For $|\lambda - \lambda_0| \leq \nu$:
  \begin{equation*}
  \texttt{eval}(\tilde{a}, \lambda - \lambda_0)^2 = \lambda.
  \end{equation*}
\end{corollary}

\subsection{Branch Selection}\label{subsec:branch_selection}

The equation $y^2 = \lambda$ has two solutions: $\pm\sqrt{\lambda}$. We show that 
the power series selects the positive branch.

\begin{lemma}[Evaluation at zero]\label{lem:eval_zero}
  \lean{l1Weighted.eval_zero}
  \leanok
  \uses{def:eval}
  For any $a \in \ell^1_\nu$:
  \begin{equation*}
  \texttt{eval}(a, 0) = a_0.
  \end{equation*}
\end{lemma}

\begin{theorem}[Positive branch identification]\label{thm:analytic_eq_sqrt}
  \lean{Example_7_7.analyticSolution_eq_sqrt}
  \leanok
  \uses{cor:analytic_is_sqrt,lem:eval_zero,def:concrete_approx}
  Suppose:
  \begin{enumerate}
    \item $\lambda_0 > 0$
    \item $\lambda > 0$
    \item $|\lambda - \lambda_0| \leq \nu$
    \item $\tilde{a}_0 > 0$
  \end{enumerate}
  Then $\texttt{eval}(\tilde{a}, \lambda - \lambda_0) = \sqrt{\lambda}$.
\end{theorem}

\begin{proof}
  \leanok
  \textbf{Step 1:} Evaluate at $z = 0$: $\texttt{eval}(\tilde{a}, 0) = \tilde{a}_0$.
  
  \textbf{Step 2:} From $\texttt{eval}(\tilde{a}, 0)^2 = \lambda_0$ and $\tilde{a}_0 > 0$: 
  $\tilde{a}_0 = \sqrt{\lambda_0} > 0$.
  
  \textbf{Step 3:} Since $\texttt{eval}(\tilde{a}, \lambda - \lambda_0)^2 = \lambda > 0$, we have 
  $\texttt{eval}(\tilde{a}, \lambda - \lambda_0) \neq 0$.
  
  \textbf{Step 4 (Positivity by IVT):} Suppose for contradiction that 
  $\texttt{eval}(\tilde{a}, \lambda - \lambda_0) < 0$. Since $\texttt{eval}(\tilde{a}, 0) > 0$ and 
  $z \mapsto \texttt{eval}(\tilde{a}, z)$ is continuous (uniform convergence of power series), 
  by the Intermediate Value Theorem there exists $c$ between $0$ and $\lambda - \lambda_0$ 
  with $\texttt{eval}(\tilde{a}, c) = 0$.
  
  But $\texttt{eval}(\tilde{a}, c)^2 = \lambda_0 + c$, so $0 = \lambda_0 + c$, giving $c = -\lambda_0 < 0$.
  
  \textit{Case 1:} If $\lambda \geq \lambda_0$, then $c \in [0, \lambda - \lambda_0]$ 
  requires $c \geq 0$, contradicting $c = -\lambda_0 < 0$.
  
  \textit{Case 2:} If $\lambda < \lambda_0$, then $c \in [\lambda - \lambda_0, 0]$ 
  requires $c \geq \lambda - \lambda_0$, i.e., $-\lambda_0 \geq \lambda - \lambda_0$, 
  so $\lambda \leq 0$, contradicting $\lambda > 0$.
  
  Therefore $\texttt{eval}(\tilde{a}, \lambda - \lambda_0) > 0$.
  
  \textbf{Step 5:} Since $\texttt{eval}(\tilde{a}, \lambda - \lambda_0) > 0$ and 
  $\texttt{eval}(\tilde{a}, \lambda - \lambda_0)^2 = \lambda$, we have 
  $\texttt{eval}(\tilde{a}, \lambda - \lambda_0) = \sqrt{\lambda}$.
\end{proof}

\subsection{Summary}\label{subsec:example_summary}

\begin{theorem}[Main result for Example 7.7]\label{thm:example_7_7_main}
  \lean{Example_7_7_Final.example_7_7_main_theorem_rat}
  \leanok
  \uses{thm:example_7_7_existence,thm:analytic_eq_sqrt}
  For $\lambda_0 = 1/3$, $\nu = 1/4$, and $N = 2$:
  \begin{enumerate}
    \item \textbf{(Existence)} There exists a unique $\tilde{a} \in \ell^1_\nu$ with 
          $\lVert \tilde{a} - \bar{a}\rVert_{\ell^1_\nu} \leq r_0$ satisfying 
          $F(\tilde{a}) = \tilde{a} \star \tilde{a} - c = 0$.
    \item \textbf{(Analytic interpretation)} The power series 
          $\tilde{x}(z) = \sum_{n=0}^\infty \tilde{a}_n z^n$ converges for $|z| \leq \nu$
          and satisfies $\tilde{x}(\lambda - \lambda_0) = \sqrt{\lambda}$ for 
          $\lambda \in [\lambda_0 - \nu, \lambda_0 + \nu]$ with $\lambda > 0$.
  \end{enumerate}
\end{theorem}

\begin{proof}
  Part (1) follows from Theorem~\ref{thm:example_7_7_existence} via the radii polynomial method.
  Part (2) follows from Theorem~\ref{thm:analytic_eq_sqrt}: the fixed-point equation 
  $\tilde{a} \star \tilde{a} = c$ implies $\tilde{x}(z)^2 = \lambda_0 + z$, and the 
  branch selection via continuity and $\tilde{a}_0 > 0$ gives the positive square root.
\end{proof}