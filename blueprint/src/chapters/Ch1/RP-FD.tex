% content.tex — dependencies for the finite-dimensional radii-polynomial theorem

\subsection*{Upstream tools (Sections 2.1–2.2)}

\begin{definition}[Closed ball]\label{dfn:Ball}
  \lean{RP.Ball}
  For $x\in\mathbb R^n$ and $r>0$, the closed ball is $B_r(x)=\{y:\|y-x\|\le r\}$.
\end{definition}

\begin{definition}[Operator norm]\label{dfn:OpNorm}
  \lean{RP.OperatorNorm}
  For a matrix $M\in M_n(\mathbb R)$, $\|M\|:=\sup_{\|v\|=1}\|Mv\|$.
\end{definition}

\begin{theorem}[Mean Value Theorem]\label{thm:MVT}
  \lean{RP.MeanValueTheorem}
  Let $g:\mathbb R^n\to\mathbb R$ be $C^1$. For all $x,y\in\mathbb R^n$ there exists $t\in(0,1)$ such that
  \[
    g(x)-g(y)=\nabla g\bigl(y+t(x-y)\bigr)\cdot (x-y).
  \]
\end{theorem}

\begin{lemma}[Mean value inequality]\label{lem:MVI}
  \lean{RP.MeanValueIneq}
  \uses{dfn:OpNorm,thm:MVT}
  Let $T:\mathbb R^n\to\mathbb R^n$ be $C^1$. Then for all $x,y$,
  \[
    \|T(x)-T(y)\|\le \sup_{0\le t\le 1}\,\|DT\bigl(y+t(x-y)\bigr)\|\,\|x-y\|.
  \]
\end{lemma}

\begin{theorem}[Contraction Mapping Theorem]\label{thm:Contraction}
  \lean{RP.ContractionMapping}
  Let $(X,d)$ be complete and $T:X\to X$ be $L$-Lipschitz with $L<1$.
  Then $T$ has a unique fixed point $x^\ast\in X$, and the iterates $x_{k+1}=T(x_k)$ converge to $x^\ast$ for every $x_0\in X$.
\end{theorem}

\subsection*{Newton-like setting and radii polynomials (Section 2.4)}

\begin{definition}[Data]\label{dfn:Data}
  \lean{RP.Data}
  Fix a norm on $\mathbb R^n$ (or $\mathbb C^n$). Let $f:\mathbb R^n\to\mathbb R^n$ be $C^1$,
  choose an approximate zero $\bar x\in\mathbb R^n$, and a matrix $A\in M_n(\mathbb R)$
  (typically $A\approx Df(\bar x)^{-1}$).
\end{definition}

\begin{definition}[Newton-like map]\label{dfn:T}
  \lean{RP.T}
  \uses{dfn:Data}
  Define $T(x)=x-Af(x)$.
\end{definition}

\begin{definition}[A priori bounds]\label{dfn:Bounds}
  \lean{RP.Bounds}
  \uses{dfn:Data,dfn:T,dfn:OpNorm}
  Choose nonnegative numbers $Y_0,Z_0$ and a nonnegative function $Z_2:(0,\infty)\to[0,\infty)$
  such that for all $r>0$ and all $c\in B_r(\bar x)$,
  \[
    \|A f(\bar x)\|\le Y_0,\qquad
    \|I-ADf(\bar x)\|\le Z_0,\qquad
    \|A\bigl(Df(c)-Df(\bar x)\bigr)\|\le Z_2(r)\,r.
  \]
\end{definition}

\begin{definition}[Radii polynomial]\label{dfn:RPoly}
  \lean{RP.RadiiPolynomial}
  \uses{dfn:Bounds}
  For $r>0$, set
  \[
    p(r):=Z_2(r)\,r^2-(1-Z_0)\,r+Y_0.
  \]
\end{definition}

\begin{lemma}[Derivative bound]\label{lem:DTbound}
  \lean{RP.DT_bound}
  \uses{dfn:T,dfn:Bounds}
  For every $c\in B_r(\bar x)$,
  \[
    DT(c)=I-ADf(c), \qquad \|DT(c)\|\le Z_0+Z_2(r)\,r =: Z(r).
  \]
\end{lemma}

\begin{lemma}[Fixed-point radii]\label{lem:FixedPointRadii}
  \lean{RP.FixedPointRadii}
  \uses{lem:MVI,lem:DTbound,dfn:Ball,thm:Contraction}
  Let $r>0$. Assume $\|T(\bar x)-\bar x\|\le Y_0$ and $\sup_{c\in B_r(\bar x)}\|DT(c)\|\le Z(r)$.
  If $(Z(r)-1)r+Y_0<0$, then $T(B_r(\bar x))\subseteq B_r(\bar x)$ and
  $T$ is a contraction on $B_r(\bar x)$ with Lipschitz constant $\le Z(r)<1$.
\end{lemma}

\begin{lemma}[Neumann-series invertibility]\label{lem:Neumann}
  \lean{RP.NeumannInvertible}
  If $\|I-B\|<1$ for a matrix $B$, then $B$ is invertible (and $B^{-1}=\sum_{k\ge 0}(I-B)^k$).
\end{lemma}

\begin{theorem}[Radii polynomials in finite dimensions]\label{thm:RadiiPolyFD}
  \lean{RP.RadiiPolynomialFiniteDim}
  \uses{dfn:RPoly,lem:DTbound,lem:FixedPointRadii,lem:Neumann,thm:Contraction}
  If there exists $r_0>0$ with $p(r_0)<0$, then:
  \begin{enumerate}
    \item[(i)] $T$ maps $B_{r_0}(\bar x)$ into itself and is a contraction there;
    \item[(ii)] there exists a unique $\tilde x\in B_{r_0}(\bar x)$ with $T(\tilde x)=\tilde x$;
    \item[(iii)] $f(\tilde x)=0$, and $\|I-ADf(\tilde x)\|<1$, hence $Df(\tilde x)$ is invertible.
  \end{enumerate}
\end{theorem}

\begin{proof}
  \leanok
  From $p(r_0)<0$ we get $Z(r_0)=Z_0+Z_2(r_0)r_0<1$ and $(Z(r_0)-1)r_0+Y_0<0$.
  Apply Lemma~\ref{lem:FixedPointRadii} with $r=r_0$ to obtain (i) and (ii); the existence and uniqueness follow from Theorem~\ref{thm:Contraction}.
  Then $T(\tilde x)=\tilde x$ implies $A f(\tilde x)=0$, hence $f(\tilde x)=0$ (since $A$ is injective);
  finally $\|I-ADf(\tilde x)\|<1$ implies $Df(\tilde x)$ is invertible by Lemma~\ref{lem:Neumann}.
\end{proof}
